\section{Background on Heart }
\begin{figure*}[htbp]
	\centering
	{
	\centering
	\subfigure[The four stages of an \acf{AP}]{
		\framebox[0.31\textwidth]{
			\begin{tikzpicture}[transform shape, xscale=0.6, yscale=0.6]
\begin{axis}
[ xlabel={Time (ms)},
ylabel={Potential ({mV})},
axis y line = left,
axis x line = bottom,
xmin=0,   xmax=300,
ymin=0,   ymax=150,
extra tick style={grid=major}
]
\addplot[color=blue!90,
mark=.,
mark size=2,
smooth,
const plot
]
table [x=t, y=v, col sep=comma] {./figures/actionPotentialData.csv};
\end{axis}

\tikzstyle{every state}=[rectangle, text centered, draw=none,text=black, draw,line width=0.3mm]

\node[state, shift={(0.55, -1.5)}, fill=green!20, minimum width=1.1cm] {RP};
\node[state, shift={(1.35, -1.5)}, fill=blue!20, minimum width=0.5cm] {UP};
\node[state, shift={(3.00, -1.5)}, fill=red!20, minimum width=2.8cm] {ERP};
\node[state, shift={(5.15, -1.5)}, fill=yellow!20, minimum width=1.5cm] {RRP};
\node[state, shift={(6.40, -1.5)}, fill=green!20, minimum width=1cm] {RP};

\end{tikzpicture}
			\label{fig:actionPotential}
		}
	} % HA
	\subfigure[\label{fig:heart} Diagram of the heart. The \acf{SA} node is the natural pacemaker of a heart]{
          \framebox[0.31\textwidth]{
				\includegraphics[width=0.30\linewidth]{figures/heart}
		} %framebox
	}% SHA
	\subfigure[An abstracted model of the conduction system as a network of $33$~nodes and conduction pathways in the Atria (Green) and Ventricles (Blue)]{ 
          \framebox[0.31\textwidth]{
          	\input{./figures/heartNetwork.tex}
          	\label{fig:heartNetwork} 
	  } %framebox
	}% code gen

}
	\caption{Electrical conduction systems of a heart}
	\label{fig:heartOverview}
\end{figure*}

In this section, using Figure~\ref{fig:heartOverview} we 
describe the electrical conduction system of the heart.

\noindent \textbf{Heart.}
The human heart (see Figure~\ref{fig:actionPotential})
 is an organ that pumps blood throughout the body.
It achieves this by regularly contracting an relaxing the muscles
which are orchestrated by a flow of electrical signals through the heart.
The signal is primarily generated automatically by the Sinoatrial~(SA) node.
First, the signal travels through left and right atrium, contracting 
the muscles and pushing the blood into ventricles.
 
Second, to ensure the both ventricles are filled, 
the Atrioventicular (AV) node introduces critical delay 
in the conduction system.   
Finally, the electrical signal propagates through 
both ventricles. This contracts the muscles and pumps the blood out.


	
\noindent \textbf{\acf{AP}.} 
The propagation of electrical signal at cellular level 
is described as a change in  voltage 
across cell membrane due to ionic flow over a period of time, 
know as \acf{AP} shown in Figure~\ref{fig:actionPotential}.  
The voltage behaviour can be described in four stages.
\begin{enumerate}
	\item \acf{RP}:
	\item \acf{UP}:
	\item \acf{ERP}:
	\item \acf{RRP}:
	
\end{enumerate} 

\noindent \textbf{Abstract model.}