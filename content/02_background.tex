\section{Background on the Heart }
\begin{figure*}[htbp]
	\centering
	{
	\centering
	\subfigure[The four stages of an \acf{AP}]{
		\framebox[0.31\textwidth]{
			\begin{tikzpicture}[transform shape, xscale=0.6, yscale=0.6]
\begin{axis}
[ xlabel={Time (ms)},
ylabel={Potential ({mV})},
axis y line = left,
axis x line = bottom,
xmin=0,   xmax=300,
ymin=0,   ymax=150,
extra tick style={grid=major}
]
\addplot[color=blue!90,
mark=.,
mark size=2,
smooth,
const plot
]
table [x=t, y=v, col sep=comma] {./figures/actionPotentialData.csv};
\end{axis}

\tikzstyle{every state}=[rectangle, text centered, draw=none,text=black, draw,line width=0.3mm]

\node[state, shift={(0.55, -1.5)}, fill=green!20, minimum width=1.1cm] {RP};
\node[state, shift={(1.35, -1.5)}, fill=blue!20, minimum width=0.5cm] {UP};
\node[state, shift={(3.00, -1.5)}, fill=red!20, minimum width=2.8cm] {ERP};
\node[state, shift={(5.15, -1.5)}, fill=yellow!20, minimum width=1.5cm] {RRP};
\node[state, shift={(6.40, -1.5)}, fill=green!20, minimum width=1cm] {RP};

\end{tikzpicture}
			\label{fig:actionPotential}
		}
	} % HA
	\subfigure[\label{fig:heart} Diagram of the heart. The \acf{SA} node is the natural pacemaker of a heart]{
          \framebox[0.31\textwidth]{
				\includegraphics[width=0.30\linewidth]{figures/heart}
		} %framebox
	}% SHA
	\subfigure[An abstracted model of the conduction system as a network of $33$~nodes and conduction pathways in the Atria (Green) and Ventricles (Blue)]{ 
          \framebox[0.31\textwidth]{
          	\input{./figures/heartNetwork.tex}
          	\label{fig:heartNetwork} 
	  } %framebox
	}% code gen

}
	\caption{Electrical conduction systems of the heart}
	\label{fig:heartOverview}
\end{figure*}

In this section, using Figure~\ref{fig:heartOverview} we 
describe the electrical conduction system of the heart.

\noindent \textbf{Heart.}
The human heart (see Figure~\ref{fig:heart})
 is an organ that pumps blood throughout the body.
It achieves this by regularly contracting and relaxing the muscles
which are orchestrated by a flow of electrical signals through the heart.
The signal is primarily generated automatically by the \ac{SA} node.
First, the signal travels through the left and right atria, contracting 
the muscles and pushing the blood into the ventricles.
 
Second, to ensure both ventricles are filled, 
the \ac{AV} node introduces critical delay 
in the conduction system.   
Finally, the electrical signal propagates through 
both ventricles. This contracts the muscles and pumps the blood out of the heart.


\noindent \textbf{\acf{AP}.} 
The propagation of electrical signals at a cellular level 
is described as a change in  voltage 
across the cell membrane due to ionic flow. Over a period of time, 
this change is known as an \acf{AP}~\cite{chen14} as shown in Figure~\ref{fig:actionPotential}. 
The \ac{AP} can be described in four stages.
\begin{enumerate}
	\item \acf{RP}: This is the steady state of the cell while awaiting 
					activation by an external stimulus.
	\item \acf{UP}: When activated by an external stimulus, 
					the cell depolarises (inward flow of positive ions) and 
					contracts the muscles. This depolarisation yields 
					a stimulus that activates neighbouring cells.
	\item \acf{ERP}: Once activated, the cell cannot be activated again 
					 due to the recovery process of the ionic channels. 
					 Any new stimulus will be blocked. 
	\item \acf{RRP}: During this period, the ionic channels are partially recovered
					 and the cell can be activated. However, the morphology
					 of the \ac{AP} will be shorter.
	
\end{enumerate}


\noindent \textbf{Abstract model.}
The human heart has over two trillion cells. For analytical purposes, a
model consisting of a network of $32$ nodes is presented in literature
and is used for testing off-the-shelf pacemakers~\cite{chen14,zhihao12}.
The abstracted electrical conduction system consisting  of 
nodes and paths is presented in Figure~\ref{fig:heartNetwork}.
The functionality of each node is described using \acf{HA} in Section~\ref{sec:HA}. The path model describes the 
conduction delay of the electrical signal.
Later in Section~\ref{sec:codeGen}, we use the abstract network model 
to illustrate our modular code generation.

 
