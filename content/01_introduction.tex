\section{Introduction}

Pacemakers are safety-critical cyber-physical devices 
that control the pacing of a heart.
In 1990-2000, close to 200,000 pacemakers have been recalled
due to software related failures~\cite{alemzadeh13}.
 One reason that these malfunctions
arise is due to the lack of proper validation.
Unlike the ideas of hardware-in-the-loop testing
  also known as \emph {emulation}~\cite{patel2015survey},
it is not possible to test on actual heart.
Further, unlike the hardware that is almost identical 
during mass production, each individual heart is different.
Thus there is a need to develop a virtual heart that is tailored
to an individual person and achieve
 personalised healthcare~\cite{Trayanova2014}. 
 This idea can be extended 
to a network of human organs, resulting in a virtual human.

Typically these biological  models are  developed by biomedical engineers.
They mimic the heart's working at the
molecular level~\cite{Trayanova2014}. 
To simulate one heart beat takes severely hours, if not days.
These models are very computationally intensive and 
are not suitable for real-time emulation,
which is required for closed-loop testing of time-critical 
controllers such as cardiac pacemakers.
Further, the models are 
 very precise resulting in sophisticated models that 
 are not amenable for  formal verification. 
Thus, there is a need to develop abstract models 
by computer systems engineers that  
(1)  \emph{capture the behaviour} of the heart (plant) 
from a pacemaker's (controller's) point of view,
allows for (2) \emph{real-time emulation} of the heart 
such that it is possible for closed-loop simulation and
is amenable for (3) \emph{formal verification} of 
functional and timing properties.


The continuous dynamics of a plant (e.g., heart) and
 the discrete behaviour of a controller (e.g., pacemaker) 
 results in so called hybrid systems. 
 These are often formally described using \acf{HA}~\cite{alur2015principles,raskin05,chen201487}.
 However, these are non-deterministic and are problematic
 for generating constructive models.
 A more abstract deterministic model will enable us to
 develop constructive/synthesisable models~\cite{Lee2014}. Also, 
 it is easier to draw trusted conclusions from simulations
 of deterministic models. Based on the
 well-known synchronous approach~\cite{benveniste03}, 
 we present a deterministic semantics for \ac{HA} and show
 constructive models in this paper.


There are many solutions for code generation from \acf{HA}...
\begin{itemize}
	\item Upenn timed automata
	\item Simulink
	\item ....
\end{itemize}
 
 \textbf{Contributions} of this paper are 
 (1) We propose a \emph{deterministic semantics} for \acf{HA}, see Section~\ref{sec:HA}.
 (2) A \emph{scalable modular code generation} approach 
 to aid real-time emulation of a plant, see Section~\ref{sec:codeGen}.
  (3) We quantitatively evaluate the efficacy of the 
  proposed approach by comparing  
   to the widely used Simulink\textsuperscript{\textregistered}
   simulation framework, see Section~\ref{sec:benchmarking}.
   
 An overview of the proposed modular code generation approach is presented
 in Figure~\ref{fig:overview}. Given a network
 of \ac{HA}, a discretised abstraction called \ac{SHA} are generated.
 Finally, executable code (captured using FSMs)  is generated for emulation.
 
 \begin{figure}[bthp]
 	\centering
 	\scalebox{0.7}{
	 % Define block styles
\tikzstyle{fileS} = [rectangle, draw, fill=red!20, 
text width=4em, text centered, minimum height=3em]
\tikzstyle{process} = [rectangle, draw, fill=blue!15, 
text width=9em, text centered, rounded corners, minimum height=4em]
\tikzstyle{line} = [draw, -latex', ultra thick]


\begin{tikzpicture}
% Place nodes

%-----------------------------------------------------------------------
\node [fileS,fill=red!10] (HA1) {$HIOA_1$ ($Def.~\ref{def:ha}$)};
\node[right of = HA1, node distance = 1.3cm](PAR){\Large $\dots$};
\node [fileS,,fill=red!10,right of = PAR, node distance = 1.3cm] (HA2) 
	{$HIOA_2$ ($Def.~\ref{def:ha}$)};	
%\node[below of = PAR, node distance=1.3cm, text width=3.7cm, text centered]
%	(TNHA){Network of HA};
%\node[draw, thick, inner sep=0.3cm, fit=(HA1) (HA2)] (NHA) {};

\node[draw, dashed, inner sep=0.3cm,
	label={[label distance=0cm]90:Section~\ref{sec:HA}}, 
	fit= (HA1) (HA2)] (S1) {};
%-----------------------------------------------------------------------


%\node [process, below  of = PAR, node distance = 2.2cm] (GSHA) {Generate FSM (step~1, Sec~\ref{sec:shaGeneration})};

%-----------------------------------------------------------------------
\node [fileS, below of = HA1, fill=red!20, node distance = 3.3cm] 
(FSM1) {$FSM_1$ ($Def.~\ref{def:ha}$)};
\node[right of = FSM1, node distance = 1.3cm](PAR){\Large $\dots$};
\node [fileS,,fill=red!20,right of = PAR, node distance = 1.3cm] (FSM2) 
{$FSM_2$ ($Def.~\ref{def:ha}$)};	

\node[draw, dashed, inner sep=0.3cm,
label={[label distance=0cm]90:Section~\ref{sec:codeGen}}, 
fit= (FSM1) (FSM2)] (S2) {};

%-----------------------------------------------------------------------
\node [fileS,,fill=blue!20, text width=8em, below of = PAR, node distance = 3.3cm] (PC) 
{$FSM_1 || FSM_2$ };
\node[ dashed, inner sep=0.1cm,
label={[label distance=0cm]90:Section~\ref{sec:composition}}, 
fit= (PC)] (S3) {};


%\node[draw, dashed, inner sep=0.2cm,
%	label={[label distance=0cm]60:Section~\ref{sec:SHA}}, 
%	fit= (GFSM)(CC)] (S3) {};

%--------------------------------------------------------------------------------
%edges
\draw [line] (HA1.south) -- (FSM1) node[midway, left]
{Step~1:  Code generation};
\draw [line] (HA2.south) -- (FSM2);
\draw [line] (FSM1) -- (FSM1|-PC.north) node[midway, left]
{Step~2:  Parallel composition};
\draw [line] (FSM2) -- (FSM2|-PC.north);
%\draw [line] (NSHA) -- (NSHA-|GFSM.west);
%\draw [line] (GFSM) -- (GFSM|-CC.south);

\end{tikzpicture}    
	}
	 \caption{Overview of the proposed modular 
	 	code generation approach \label{fig:overview}}
\end{figure}
      
 
 
