\section{Introduction}

Pacemakers are safety-critical \acp{CPS} 
that control the pacing of a heart.
In 1990-2000, close to 200,000 pacemakers have been recalled
due to software related failures~\cite{alemzadeh13}.
 One reason that these malfunctions
arise is due to the lack of proper validation.
Unlike the ideas of hardware-in-the-loop testing,
also known as \emph {emulation}~\cite{patel2015survey},
it is not possible to test on an actual heart.
Further, unlike the hardware which is almost identical 
during mass production, each individual heart is different.
Thus there is a need to develop a virtual heart that is tailored
to an individual person and achieve
personalised healthcare~\cite{Trayanova2014}. 
This idea can be extended 
to a network of human organs, resulting in a virtual human.

Typically these biological  models are  developed by biomedical engineers.
They mimic the heart's working at the
molecular level~\cite{Trayanova2014}. 
To simulate one heart beat takes several hours, if not days.
These models are very computationally intensive and 
are not suitable for real-time emulation,
which is required for closed-loop testing of time-critical 
controllers such as cardiac pacemakers.
Further, the models are 
 very complex and hence 
 are not amenable for  formal verification. 
Thus, there is a need to develop abstract models 
by computer systems engineers that  
(1)  \emph{capture the behaviour} of the heart (plant) 
from a pacemaker's (controller's) point of view,
(2) \emph{allows for real-time emulation} of the heart 
such that it is possible for closed-loop simulation and
(3) \emph{are amenable for formal verification} of 
functional and timing properties.


The continuous dynamics of a plant (e.g. heart) and
 the discrete behaviour of a controller (e.g. pacemaker) 
 result in so called hybrid systems. 
 These are often formally described using \acf{HA}~\cite{alur2015principles,raskin05,chen201487}.
 However, these are non-deterministic and are problematic
 for generating constructive models.
 A more abstract deterministic model will enable us to
 develop constructive/synthesizable models~\cite{Lee2014}. Also, 
 it is easier to draw trusted conclusions from simulations
 of deterministic models. Based on the
 well-known synchronous approach~\cite{benveniste03}, 
 we present a deterministic semantics for \ac{HA} and show
 constructive models in this paper.


Traditional \ac{CPS} design uses commercial tools such as \simulink for plant simulation and emulation.
However, such designs do not preserve formal semantics of the models and are more verbose to describe.
Academic tools for code generation from \ac{HA} models such as Ptolemy~\cite{ptolemaeus2014system} and Z\'{e}lus~\cite{bourke13zelus}.
While these tools preserve the formal semantics, their reliance on dynamic numerical solvers makes them unsuitable for real-time emulation of plants.
Such tools are capable of expressing the full non-deterministic nature of \acp{HA} rather than the deterministic subset described here.

   
 An overview of the proposed modular code generation approach is presented
 in Figure~\ref{fig:overview}. Given a network
 of \acp{HA}, a numerical abstraction called \ac{SHA} are generated.
 Finally, executable code (captured using \acp{FSM}) is generated for emulation.
 
 \begin{figure}[bthp]
 	\centering
 	\scalebox{0.7}{
	 % Define block styles
\tikzstyle{decision} = [diamond, draw, fill=white!30, 
text width=5em, text badly centered, node distance=3.2cm, inner sep=0pt]
\tikzstyle{file} = [rectangle, draw, fill=gray!20, 
text width=5em, text centered, minimum height=4em]
\tikzstyle{process} = [rectangle, draw, fill=blue!15, 
text width=5em, text centered, rounded corners, minimum height=4em]
\tikzstyle{line} = [draw, -latex', ultra thick]
\tikzstyle{state} = [draw, ellipse,fill=red!30, 
node distance=3.5cm, text width=5em, 
text badly centered,
minimum height=2em]


\begin{tikzpicture}[node distance = 1cm, auto]
% Place nodes
\node [file] (HA) {{HA}};

\node [decision, right of = HA] (D) {Is a WHA~? (step 1,a
	Sec~\ref{sec:static-analysis-ha})};

\node [state, right of = D] (E) {Invalid input};

\node [file, right of = E, node distance=3cm] (FSM) {FSM/ C-code};


\node [process, below of = D, node distance =3cm] (GSHA) {Generate
	\ac{SHA} (step 2, Sec~\ref{sec:generation-sha})};

\node [file, below of = E, node distance =3cm] 
(SHA) {\ac{SHA}};

\node [process, below of = FSM, node distance =3cm] (GFSM)
{Generate backend code (step 3,
	Sec~\ref{sec:back-code-gener})};



%edges
\path [line] (HA) -- (D);
\path [line] (D)-- node[near start]{no}(E);
\path [line] (D)-- node[near start]{yes}(GSHA);

\path [line] (GSHA) -- (SHA);
\path [line] (SHA) -- (GFSM);
\path [line] (GFSM) -- (FSM);
\end{tikzpicture}    
	}
	 \caption{Overview of the proposed modular 
	 	code generation approach \label{fig:overview}}
\end{figure}
      
\textbf{Contributions} of this paper are 
(1) We propose a \emph{deterministic semantics} for \acf{HA}, see Section~\ref{sec:HA}.
(2) A \emph{scalable modular code generation} approach 
to aid real-time emulation of a plant, see Section~\ref{sec:codeGen}.
(3) We \emph{quantitatively evaluate} the efficacy of the 
proposed approach through comparisons  
to the widely used \simulink
simulation framework, see Section~\ref{sec:benchmarking}.
 
