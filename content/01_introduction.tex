\section{Introduction}

Pacemakers are safety-critical \acp{CPS} that control the pacing of a
heart for providing therapy for bradycardia -- abnormally slow pacing of
the heart.  Such devices must operate in a fail-safe manner all the
time. However, between 1990-2000, close to 200,000 pacemakers have been
recalled due to software related
failures~\cite{alemzadeh13}. Considering this, there is a need for the
development of better processes for validation and certification of such
devices. We propose the well known, and widely used, engineering
technique of \emph{emulation}~\cite{patel2015survey} to tackle this
problem. Emulation, also known as hardware-in-the-loop simulation, is
used to validate controllers (such as motor controllers) by running them
in closed-loop with the actual plant (e.g., the synchronous motor).
However, for emulation of pacemakers, the use of the actual plant
(i.e. animal / human organs) is limiting.  Hence, there is a need for
the development of high-fidelity heart models that can provide the
required real-time response to facilitate emulation. Bioengineering
heart models~\cite{Trayanova2014} provide excellent model fidelity at
the expense of computation time as the simulation of a single heart beat
may take several hours. Hence, such models are not suitable for
emulation.

Recently, timed automata~\cite{zhihao12} based heart models have been
developed primarily for model checking. These models abstract the
continuous dynamics and hence, are unsuitable for emulation.  In
contrast to this, \acf{HIOA}~\cite{alur2015principles, raskin05} is used
for modelling the forward conduction system of the heart using 33 nodes
in~\cite{chen14}. This work is the starting point for real-time
emulation by demonstrating that \simulink can be used for closed-loop
verification of pacemakers. However, this work has limited model
fidelity and the limitations of the tool \simulink are inherited by the
developed approach.  \simulink has semantic limitations, as the
semantics of composition is unclear~\cite{need-some-ref}.  Moreover,
there is no direct correspondence between the \simulink model and the
\ac{HIOA} models. Finally, \simulink generated code has scalability
issues, due to which the researchers modelled a 33-node conduction
system.

%We develop an approach by combining node models (which are extensions to
%the models developed by~\cite{chen14}) with \emph{path models} (in timed
%automata) to accurately model the conduction delay between nodes. We
%are, thus, able to model the re-entrant behaviour of the heart.

{\color{red}[PARA. Needs to be updated]} 
For modular code generation, we have developed an approach based on the
well known synchronous languages~\cite{benveniste03}.  We formalise a
 subset of \ac{HIOA} called Synchronous
Hybrid Input Output Automata \ac{SHIOA}. Using the concept of delayed
synchronous composition~\cite{boussinot96}, we are then able to produce
modular code for each node separately. The generated code is both
smaller and faster execution times, essential for emulation. In
addition, such code is based on semantic-preserving code generation,
similar in spirit to Ptolemy~\cite{ptolemaeus2014system} and
Z\'{e}lus~\cite{bourke13zelus}.  However, unlike these approaches, the
presented approach does not depend upon dynamic numerical solvers, which
are not ideal for the \textit{emulation} of the heart.

\ignore{One reason that these malfunctions arise is due to the lack of
  proper validation.  Unlike the ideas of hardware-in-the-loop testing,
  also known as \emph {emulation}~\cite{patel2015survey}, it is not
  possible to test on an actual heart.  Further, unlike the hardware
  which is almost identical during mass production, each individual
  heart is different.  Thus there is a need to develop a virtual heart
  that is tailored to an individual person and achieve personalised
  healthcare~\cite{Trayanova2014}.  This idea can be extended to a
  network of human organs, resulting in a virtual human.

  Typically these biological models are developed by biomedical
  engineers.  They mimic the heart's working at the molecular
  level~\cite{Trayanova2014}.  To simulate one heart beat takes several
  hours, if not days.  These models are very computationally intensive
  and are not suitable for real-time emulation, which is required for
  closed-loop testing of time-critical controllers such as cardiac
  pacemakers.  Further, the models are very complex and hence are not
  amenable for formal verification.  Thus, there is a need to develop
  abstract models by computer systems engineers that (1) \emph{capture
    the behaviour} of the heart (plant) from a pacemaker's
  (controller's) point of view, (2) \emph{allows for real-time
    emulation} of the heart such that it is possible for closed-loop
  simulation and (3) \emph{are amenable for formal verification} of
  functional and timing properties.


  The continuous dynamics of a plant (e.g. heart) and the discrete
  behaviour of a controller (e.g. pacemaker) result in so called hybrid
  systems.  These are often formally described using
  \acf{HIOA}~\cite{alur2015principles,raskin05,chen201487}.  However,
  these are non-deterministic and are problematic for generating
  constructive models.  A more abstract deterministic model will enable
  us to develop constructive/synthesizable models~\cite{Lee2014}. Also,
  it is easier to draw trusted conclusions from simulations of
  deterministic models. Based on the well-known synchronous
  approach~\cite{benveniste03}, we present a deterministic semantics for
  \ac{HIOA} and show constructive models in this paper.


  Traditional \ac{CPS} design uses commercial tools such as \simulink
  for plant simulation and emulation.  However, such designs do not
  preserve formal semantics of the models and are more verbose to
  describe.  Academic tools for code generation from \ac{HIOA} models such
  as Ptolemy~\cite{ptolemaeus2014system} and
  Z\'{e}lus~\cite{bourke13zelus}.  While these tools preserve the formal
  semantics, their reliance on dynamic numerical solvers makes them
  unsuitable for real-time emulation of plants.  Such tools are capable
  of expressing the full non-deterministic nature of \acp{HIOA} rather
  than the deterministic subset described here.}

   
An overview of the proposed approach is presented in
Figure~\ref{fig:overview}. It has two steps.
(1) Given a network of \acp{HIOA} for the heart nodes,
in Section~\ref{sec:codeGen}, 
 the corresponding executable code (captured using
 \acp{FSM} for each node) is generated.
 (2) Then, in Section~\ref{sec:composition}, parallel composition of 
 \acp{FSM} is presented.
%\ac{SHIOA} is generated. \ac{SHIOA} is an 
% abstraction using the synchronous approach
%to facilitate the generation of 


\begin{figure}[bthp]
  \centering
  \scalebox{0.7}{
    % Define block styles
\tikzstyle{fileS} = [rectangle, draw, fill=red!20, 
text width=4em, text centered, minimum height=3em]
\tikzstyle{process} = [rectangle, draw, fill=blue!15, 
text width=9em, text centered, rounded corners, minimum height=4em]
\tikzstyle{line} = [draw, -latex', ultra thick]


\begin{tikzpicture}
% Place nodes

%-----------------------------------------------------------------------
\node [fileS,fill=red!10] (HA1) {$HIOA_1$ ($Def.~\ref{def:ha}$)};
\node[right of = HA1, node distance = 1.3cm](PAR){\Large $\dots$};
\node [fileS,,fill=red!10,right of = PAR, node distance = 1.3cm] (HA2) 
	{$HIOA_2$ ($Def.~\ref{def:ha}$)};	
%\node[below of = PAR, node distance=1.3cm, text width=3.7cm, text centered]
%	(TNHA){Network of HA};
%\node[draw, thick, inner sep=0.3cm, fit=(HA1) (HA2)] (NHA) {};

\node[draw, dashed, inner sep=0.3cm,
	label={[label distance=0cm]90:Section~\ref{sec:HA}}, 
	fit= (HA1) (HA2)] (S1) {};
%-----------------------------------------------------------------------


%\node [process, below  of = PAR, node distance = 2.2cm] (GSHA) {Generate FSM (step~1, Sec~\ref{sec:shaGeneration})};

%-----------------------------------------------------------------------
\node [fileS, below of = HA1, fill=red!20, node distance = 3.3cm] 
(FSM1) {$FSM_1$ ($Def.~\ref{def:ha}$)};
\node[right of = FSM1, node distance = 1.3cm](PAR){\Large $\dots$};
\node [fileS,,fill=red!20,right of = PAR, node distance = 1.3cm] (FSM2) 
{$FSM_2$ ($Def.~\ref{def:ha}$)};	

\node[draw, dashed, inner sep=0.3cm,
label={[label distance=0cm]90:Section~\ref{sec:codeGen}}, 
fit= (FSM1) (FSM2)] (S2) {};

%-----------------------------------------------------------------------
\node [fileS,,fill=blue!20, text width=8em, below of = PAR, node distance = 3.3cm] (PC) 
{$FSM_1 || FSM_2$ };
\node[ dashed, inner sep=0.1cm,
label={[label distance=0cm]90:Section~\ref{sec:composition}}, 
fit= (PC)] (S3) {};


%\node[draw, dashed, inner sep=0.2cm,
%	label={[label distance=0cm]60:Section~\ref{sec:SHA}}, 
%	fit= (GFSM)(CC)] (S3) {};

%--------------------------------------------------------------------------------
%edges
\draw [line] (HA1.south) -- (FSM1) node[midway, left]
{Step~1:  Code generation};
\draw [line] (HA2.south) -- (FSM2);
\draw [line] (FSM1) -- (FSM1|-PC.north) node[midway, left]
{Step~2:  Parallel composition};
\draw [line] (FSM2) -- (FSM2|-PC.north);
%\draw [line] (NSHA) -- (NSHA-|GFSM.west);
%\draw [line] (GFSM) -- (GFSM|-CC.south);

\end{tikzpicture}    
  }
  \caption{Overview of the proposed modular 
    code generation approach \label{fig:overview}}
\end{figure}
      
\textbf{Contributions} of the paper are: (1) We propose a
new definition for \acf{HIOA} (see
Section~\ref{sec:HA}) to capture  the electrical conduction system of
the heart. (2) We have developed a \emph{modular code generation}
approach, for the first time for \ac{HIOA} models
(Section~\ref{sec:codeGen}), which is also semantic preserving.  The
proposed approach performs code generation using a new synchronous
semantics~\cite{benveniste03} of \ac{HIOA}. More importantly, the
developed semantics is not restrictive, unlike~\cite{alur2003generating,
  kim2003modular} and does not interact with dynamic numerical solvers
unlike~\cite{ptolemaeus2014system, bourke13zelus}.  (3) We
\emph{quantitatively evaluate} the efficacy of the proposed approach
relative \simulink (Section~\ref{sec:benchmarking}) to demonstrate the
scalability of the approach relative to the best known model in
published literature~\cite{chen14} to show that code generation is
feasible for thousands of nodes of the heart, unlike the 32-node
published model. We have also modelled a much more realistic human heart
as the modelled system can mimic its re-entrant behaviour.
 

 

%%% Local Variables:
%%% mode: latex
%%% TeX-master: "../DATE2016_codegen"
%%% End:
