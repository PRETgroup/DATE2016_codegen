\section{Modular code generation}
\label{sec:codeGen}

\begin{figure}[t!]
	\centering
	\scalebox{0.7}{
		% Define block styles
\tikzstyle{file} = [rectangle, draw, fill=gray!20, 
    text width=7em, text centered, minimum height=4em]
\tikzstyle{process} = [rectangle, draw, fill=blue!15, 
    text width=7em, text centered, rounded corners, minimum height=4em]
\tikzstyle{line} = [draw, -latex', ultra thick]
    
    
\begin{tikzpicture}
    % Place nodes
    %\node [file] (HA) {Network of \ac{HA}};
    
    %\node [process, below of = HA, node distance = 3cm] (GSHA) {Generate network of \ac{SHA} (step 1, Sec~\ref{sec:shaGeneration})};
    
    \node [file, right of = GSHA, node distance = 4cm] (NSHA) {Network of \ac{SHA}};
    
    \node [process, right of = NSHA, node distance = 4cm] (GFSM) {Generate backend code (step 2, Sec~\ref{sec:backendCodeGeneration})};
		 
	\node [file, right of = GFSM, node distance = 3cm] (FSM) {FSM/ C-code};
		 
		 
	%edges
%	\path [line] (HA) -- (GSHA);
%	\path [line] (GSHA) -- (SHA);
	\path [line] (NSHA) -- (GFSM);
	\path [line] (GFSM) -- (FSM);
\end{tikzpicture}    
	}
	\caption{Process for compiling a network of \acf{HA}}
	\label{fig:compilingHA}
\end{figure}

%\begin{figure}[htbp]
%	\centering
%	{
	\centering
	\subfigure[\acf{HA}, see Definition~\ref{def:ha}. Reproduced from Figure~\ref{fig:waterTankHAtank}. Symbols $K$ and $h$ are constants with values $0.075$ and $150$, respectively. \label{fig:codeGenHA}
	]{
	\framebox[0.50\textwidth]{
		\begin{tikzpicture}[->,>=stealth',shorten >=1pt,auto,
			node distance=5.4cm,
			semithick,scale=0.9, transform shape]
			\tikzstyle{every state}=[rectangle,rounded corners,
			minimum height = 1.2cm, text width=2.2cm, text centered, fill=blue!20,draw=none,text=black, draw,line width=0.3mm]
			
			\node[state, 
			label={[shift={(0,-1.9)}]$ x = 20$}, 
			label={[shift={(-1.5,-0.2)}]\large $ \mathbf{t_1}$ }]
			(T1)  {$\dot{x}=0$};
			
			\draw[<-, dashed](T1.180) -- node[below]
                        {initial} ++(-1.0cm,-0cm);
			
			\node[state, 
			label={[shift={(0,0.1)}]$20\leq x \leq 100$}, 
			label={[shift={(-1.5,-0.2)}]\large $\mathbf{t_2}$   }]
			(T2) [node distance=3.4cm, above of=T1] 	 {$\dot{x}=K(h-x)$};
			
			\node[state, 
			label={[shift={(0,0.1)}]$ x = 100$}, 
			label={[shift={(1.5,-0.2)}]\large $\mathbf{t_3}$  }] 
			(T3) [right of=T2] 	 {$\dot{x}=0$};
			
			\node[state, 
			label={[shift={(0,-1.9)}]$20\leq x \leq 100$}, 
			label={[shift={(1.5,-0.2)}]\large $\mathbf{t_4}$  }]
			(T4) [right of=T1] 	 {$\dot{x}=-Kx$};
			
			
			\draw (T1) -- (T2) node [midway] 
			{\footnotesize $\inSignal{ON},x^\prime = x$};
			\draw (T2) -- (T3) node [midway] 
			{\footnotesize $\iSignal,x=100 \wedge x^\prime = x$};
			\draw (T3) -- (T4) node [midway, right] 
			{\footnotesize $\inSignal{OFF},x^\prime = x$};
			\draw (T4) -- (T1) node [midway] 
			{\footnotesize $\iSignal,x=20 \wedge x^\prime = x$};
			
			%diagonal arrows
			\draw (T2.335) -- (T4) node [midway, sloped] 
			{\footnotesize $\inSignal{OFF},x^\prime = x$};
			\draw (T4.155) -- (T2) node [midway, sloped] 
			{\footnotesize $\inSignal{ON},x^\prime = x$};
		\end{tikzpicture}
	  } %framebox
	} % HA
	
	\subfigure[\label{fig:codeGenSHA} \acf{SHA}, see
        Definition~\ref{def:sha}. Flow predicates are described using
        witness functions. Values of the constants $h$ and $K$ are $150$
        and $0.075$, respectively. Furthermore, witness function
        $F_1=C_1 e^{-0.075\times t} + 150.0$ and
        $F_2=C_1 e^{-0.075\times t}$ ]{
          \framebox[0.5\textwidth]{
		\begin{tikzpicture}[->,>=stealth',shorten >=1pt,auto,
		node distance=5.7cm,
		semithick,scale=0.9, transform shape]
		\tikzstyle{every state}=[rectangle,rounded corners,
		minimum height = 1.2cm, text width=2.9cm, text centered, fill=blue!20,draw=none,text=black, draw,line width=0.3mm]
		
		\node[state, 
		label={[shift={(0,-1.9)}]$ x = 20$}, 
		label={[shift={(-1.5,0)}]\large $ \mathbf{t_1}$ }]
		(T1)  { $\mathbf{x(t)=20}$ $\mathbf{x(0) \in [20,20]}$};
		
		
		\draw[<-, dashed](T1.180) -- node[below] {initial}
                ++(-1cm,0cm);
		
		
		
		\node[state, 
		label={[shift={(0,0.1)}]$20\leq x \leq 100$}, 
		label={[shift={(-1.5,0)}]\large $\mathbf{t_2}$   }]
		(T2) [node distance=3.4cm, above of=T1]  	 {\small  $\mathbf{x(t)=F_1(t,C_1)}$  $\mathbf{x(0) \in [20,100]}$};
		
		\node[state, 
		label={[shift={(0,0.1)}]$ x = 100$}, 
		label={[shift={(1.5,0)}]\large $\mathbf{t_3}$  }] 
		(T3) [right of=T2] 	 {$\mathbf{x(t)=100}$  $\mathbf{x(0) \in [100,100]}$};
		
		\node[state, 
		label={[shift={(0,-1.9)}]$20\leq x \leq 100$}, 
		label={[shift={(1.5,0)}]\large $\mathbf{t_4}$  }]
		(T4) [right of=T1] 	 {\small  $\mathbf{x(t)=F_2(t,C_1)}$  
			$\mathbf{x(0) \in [20,100]}$};
		
		
		\draw (T1) -- (T2) node [midway] 
		{\footnotesize $ON,x^\prime = x$};
		\draw (T2) -- (T3) node [midway] 
		{\footnotesize $\iSignal,x=100 \wedge x^\prime = x$};
		\draw (T3) -- (T4) node [midway, right] 
		{\footnotesize $OFF,x^\prime = x$};
		\draw (T4) -- (T1) node [midway] 
		{\footnotesize $\iSignal,x=20 \wedge x^\prime = x$};
		
		
		\draw (T2.335) -- (T4) node [midway, sloped] 
		{\footnotesize $OFF,x^\prime = x$};
		\draw (T4.155) -- (T2) node [midway, sloped] 
		{\footnotesize $ON,x^\prime = x$};
		
		\end{tikzpicture}
		} %framebox
	}% SHA
	
	\subfigure[\label{fig:codeGenDTTS} {Synchronous Witness Automata
          (SWA). We abuse the notation $x[k]$ to update the value of
          $x$, although $x[k]$ represents the valuation of the
          continuous variable $x$. This physical time
          $t=k \times \delta$, where $k$ is the logical tick and
          $\delta$ is the tick length.}]{ \framebox[0.49\textwidth]{
		\begin{tikzpicture}[->,>=stealth',shorten >=1pt,auto,node distance=5.6cm,
			semithick,scale=0.85, transform shape]
			\tikzstyle{every state}=[circle,rounded corners,
                        minimum height = 1.2cm, text width=1.2cm, text
                        centered, fill=blue!20,draw=none,text=black,
                        draw,line width=0.3mm]
			
			\node[state] (T1)  { $t_1$ };
			
			\draw[<-, dashed](T1.180) -- node[above, text
                        width =1.5cm] {initial $x[0]=20,$ $k=0$}
                        ++(-1.3cm,0cm);
			
			\node[state]
			(T2) [above of=T1] 	 {$t_2$};
			
			\node[state] 
			(T3) [right of=T2] 	 {$t_3$};
			
			\node[state] 
			(T4) [right of=T1] 	 { $t_4$};
			
			
			%edges
			\draw (T1) -- (T2) node [pos=0.5, rotate =90, above] 
			{ $\frac{{ON} \wedge \overline{OFF} \wedge x[k]=20}{C_1=x[k]-150, x[0]=x[k],k=0}$};
			
			\draw (T2) -- (T3) node [midway] 
			{ $\frac{\overline{ON} \wedge \overline{OFF} \wedge x[k]=100}{x[0]=x[k], k=0}$};
			
			\draw (T4) -- (T1) node [midway] 
			{ $\frac{\overline{ON}
					 \wedge \overline{OFF} \wedge x[k]=20}
				{x[0]=x[k],k=0}
				$};
			
			\draw (T3) -- (T4) node [midway,rotate =90, below] 
			{$\frac{ \overline{ON} \wedge {OFF} \wedge x[k]=100}
				{C_1=x[k], x[0]=x[k],k=0} $};
			
			%\draw (T2.180)--([xshift=-1em]T2.180)|-([yshift=1em]T2.90)--(T2.90) node [pos=0.5, rotate =0, above]
			
			\draw (T2.180) to[out=180,in=90, distance=1.5cm] (T2.90);
			\draw (T2)  node[yshift=1.7cm]
			{ $\frac{\overline{ON}  \wedge \overline{OFF} \wedge (20 \leq x[k]  \leq 100)} {x[k+1]=F_1(\delta,k,C_1), k=k+1}   $ };
			
			\draw (T3.0) to[out=0,in=90, distance=1.5cm] (T3.90);
			\draw (T3)  node[yshift=1.7cm]
			{ $\frac{\overline{ON}  \wedge \overline{OFF} \wedge ( x[k]  = 100)} {x[k+1]=x[k],k=k+1}   $ };
			
			\draw (T4.0) to[out=0,in=-90, distance=1.5cm] (T4.-90);
			\draw (T4)  node[yshift=-1.7cm]
			{ $\frac{\overline{ON}  \wedge \overline{OFF} \wedge ( x[k]  = 100)}{x[k+1]=F_2(\delta,k,C_1),k=k+1}   $ };
			
			\draw (T1.-90) to[out=-90,in=200, distance=1.5cm] (T1.200);
			\draw (T1)  node[yshift=-1.7cm]
			{ $\frac{\overline{ON}  \wedge \overline{OFF} \wedge ( x[k]  = 20)}{x[k+1]=x[k],k=k+1}   $ };
			
			%diagnonal arrows
			\draw (T2) to[out=-20,in=110, distance=1.5cm] (T4);
			\draw (T4)  node[yshift=3.7cm, xshift=-2cm, rotate =-45]
			{ $\frac{\overline{ON}  \wedge {OFF} }{C_1=x[k], x[0]=x[k],k=0}   $ };
			
			\draw (T4) to[out=160,in=-70, distance=1.5cm] (T2);
			\draw (T2)  node[yshift=-3.7cm, xshift=2cm, rotate =-45]
			{ $\frac{{ON}  \wedge \overline{OFF} }{C_1=x[k]-150, x[0]=x[k],k=0}   $ };
			
		\end{tikzpicture}
	  } %framebox
	}% code gen

}
%	\caption{The water tank component from the running example}
%	\label{fig:codeGenOverview}
%\end{figure}

In this section we present our approach for modular code generation from
a network of \acp{HA}. 
An overview of the solution is presented in Figure~\ref{fig:compilingHA} and the two steps are outlined in detail below in Sections \ref{sec:shaGeneration} and \ref{sec:backendCodeGeneration}.

\subsection{Generation of a network of \ac{SHA}}
\label{sec:shaGeneration}

This section describes step 1 of the compilation process from Figure~\ref{fig:compilingHA}.
This step involves converting all of the \acp{ODE} in each of the \acp{HA} into their forward Euler equivalent methods.

Given the \ac{ODE} for $v_x$ in state $\mathbf{q_1}$ of the Heart Cell \ac{HA} defined in Figure~\ref{fig:cellHA} (reproduced in Equation~\ref{eq:ode}), the forward Euler method is applied on each iteration in order to calculate $v_x$ numerically.
The forward Euler equivalent for this ODE is shown in Equation~\ref{eq:euler_equiv} where $\delta$ is the time-step size and $v^\prime_x$ represents the value of $v_x$ after the current iteration.

\begin{equation}
\dot{v_x} = a^1_x v_x + \beta_x g(\vec{v})
\label{eq:ode}
\end{equation}

\begin{equation}
v^\prime_x = v_x + \delta * (a^1_x v_x + \beta_x g(\vec{v}))
\label{eq:euler_equiv}
\end{equation}

\renewcommand{\algorithmiccomment}[1]{// #1}
\renewcommand{\algorithmicrequire}{\textbf{Input:}}
\renewcommand{\algorithmicensure}{\textbf{Output:}}
\begin{algorithm}[t!]
	\begin{algorithmic}[1]
		\REQUIRE Network of HA has 
		\ENSURE Network of SHA shas
		\FORALL{$ha \in has$} \label{alg:HAsToSHAs:allHAs}
			\FORALL{$loc \in ha$}  \label{alg:HAsToSHAs:allLocs}
				\FORALL{$ode \in loc$} \label{alg:HAsToSHAs:allODEs}
					\STATE $eq \leftarrow create\_euler(ode)$ \label{alg:HAsToSHAs:createEuler}
					\STATE $ode \leftarrow eq$ \label{alg:HAsToSHAs:assignOde}
				\ENDFOR
			\ENDFOR
		\ENDFOR
		\RETURN this
	\end{algorithmic}
	\caption{The algorithm to generate a Network of \acp{SHA} from a Network of \acp{HA}}
	\label{alg:HAsToSHAs}
\end{algorithm}

The procedure for performing this conversion across the entire network of \acp{HA} to generate a network of \acp{SHA} is shown in Algorithm~\ref{alg:HAsToSHAs}.
The algorithm needs to visit every \ac{HA} in the network, every location within each \ac{HA}, and every \ac{ODE} in each location (lines~\ref{alg:HAsToSHAs:allHAs} - \ref{alg:HAsToSHAs:allODEs}).
For each of these \acp{ODE} the forward Euler equivalent is created in the same manner as described earlier (line~\ref{alg:HAsToSHAs:createEuler}).
Each \ac{ODE} is eventually replaced (line~\ref{alg:HAsToSHAs:assignOde}) by its equivalent numerical solution in order to form the network of \acp{SHA}.

\subsection{Backend code generation}
\label{sec:backendCodeGeneration}

\begin{itemize}
	\item Overview Figure
	\item code generation
	\item composition
	\item Proof for Determinism (still holds ?)
\end{itemize}