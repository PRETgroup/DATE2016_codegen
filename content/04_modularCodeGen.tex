\section{Modular code generation}
\label{sec:codeGen}

\begin{figure}[t!]
	\centering
	\scalebox{0.7}{
		% Define block styles
\tikzstyle{file} = [rectangle, draw, fill=gray!20, 
    text width=7em, text centered, minimum height=4em]
\tikzstyle{process} = [rectangle, draw, fill=blue!15, 
    text width=7em, text centered, rounded corners, minimum height=4em]
\tikzstyle{line} = [draw, -latex', ultra thick]
    
    
\begin{tikzpicture}
    % Place nodes
    \node [file] (HA) {Network of \ac{HA}};
    
    \node [process, below of = HA, node distance = 3cm] (GSHA) {Generate network of \ac{SHA} (step 1, Sec~\ref{sec:shaGeneration})};
    
    \node [file, right of = GSHA, node distance = 4cm] (SHA) {Network of \ac{SHA}};
    
    \node [process, right of = SHA, node distance = 4cm] (GFSM) {Generate backend code (step 2, Sec~\ref{sec:backendCodeGeneration})};
		 
	\node [file, above of = GFSM, node distance = 3cm] (FSM) {FSM/ C-code};
		 
		 
	%edges
	\path [line] (HA) -- (GSHA);
	\path [line] (GSHA) -- (SHA);
	\path [line] (SHA) -- (GFSM);
	\path [line] (GFSM) -- (FSM);
\end{tikzpicture}    
	}
	\caption{Process for compiling a network of \acf{HA}}
	\label{fig:compilingHA}
\end{figure}

%\begin{figure}[htbp]
%	\centering
%	\input{./figures/codeGenOverview}
%	\caption{The water tank component from the running example}
%	\label{fig:codeGenOverview}
%\end{figure}

In this section we present our approach for modular code generation from
a network of \acp{HA}. 
An overview of the solution is presented in Figure~\ref{fig:compilingHA} and the two steps are outlined in detail below in Sections \ref{sec:shaGeneration} and \ref{sec:backendCodeGeneration}.

\subsection{Generation of a network of \ac{SHA}}
\label{sec:shaGeneration}

This section describes step 1 of the compilation process from Figure~\ref{fig:compilingHA}.
This step involves converting all of the \acp{ODE} in each of the \acp{HA} into their forward Euler equivalent methods.

Given the \ac{ODE} for $v_x$ in state $\mathbf{q_1}$ of the Heart Cell \ac{HA} defined in Figure~\ref{fig:cellHA} (reproduced in Equation~\ref{eq:ode}), the forward Euler method is applied on each iteration in order to calculate $v_x$ numerically.
The forward Euler equivalent for this ODE is shown in Equation~\ref{eq:euler_equiv} where $\delta$ is the time-step size and $v^\prime_x$ represents the value of $v_x$ after the current iteration.

\begin{equation}
\dot{v_x} = a^1_x v_x + \beta_x g(\vec{v})
\label{eq:ode}
\end{equation}

\begin{equation}
v^\prime_x = v_x + \delta * (a^1_x v_x + \beta_x g(\vec{v}))
\label{eq:euler_equiv}
\end{equation}

\renewcommand{\algorithmiccomment}[1]{// #1}
\renewcommand{\algorithmicrequire}{\textbf{Input:}}
\renewcommand{\algorithmicensure}{\textbf{Output:}}
\begin{algorithm}[t!]
	\begin{algorithmic}[1]
		\REQUIRE Network of HA has 
		\ENSURE Network of SHA shas
		\FORALL{$ha \in has$} \label{alg:HAsToSHAs:allHAs}
			\FORALL{$loc \in ha$}  \label{alg:HAsToSHAs:allLocs}
				\FORALL{$ode \in loc$} \label{alg:HAsToSHAs:allODEs}
					\STATE $eq \leftarrow create\_euler(ode)$ \label{alg:HAsToSHAs:createEuler}
					\STATE $ode \leftarrow eq$ \label{alg:HAsToSHAs:assignOde}
				\ENDFOR
			\ENDFOR
		\ENDFOR
		\RETURN this
	\end{algorithmic}
	\caption{The algorithm to generate a Network of \acp{SHA} from a Network of \acp{HA}}
	\label{alg:HAsToSHAs}
\end{algorithm}

The procedure for performing this conversion across the entire network of \acp{HA} to generate a network of \acp{SHA} is shown in Algorithm~\ref{alg:HAsToSHAs}.
The algorithm needs to visit every \ac{HA} in the network, every location within each \ac{HA}, and every \ac{ODE} in each location (lines~\ref{alg:HAsToSHAs:allHAs} - \ref{alg:HAsToSHAs:allODEs}).
For each of these \acp{ODE} the forward Euler equivalent is created in the same manner as described earlier (line~\ref{alg:HAsToSHAs:createEuler}).
Each \ac{ODE} is eventually replaced (line~\ref{alg:HAsToSHAs:assignOde}) by its equivalent numerical solution in order to form the network of \acp{SHA}.

\subsection{Backend code generation}
\label{sec:backendCodeGeneration}

\begin{itemize}
	\item Overview Figure
	\item code generation
	\item composition
	\item Proof for Determinism (still holds ?)
\end{itemize}