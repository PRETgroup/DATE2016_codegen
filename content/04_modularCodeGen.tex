\section{Modular code generation}
\label{sec:codeGen}

In this section we present our approach for modular code generation from
a network of \acp{HIOA} as it is implemented in our tool, called
\ourTool. It is a two step procedure as presented previously in
Figure~\ref{fig:overview}. Step 1 generates a \ac{FSM} representation
for each \ac{HIOA} in the network. In the resultant \ac{FSM} all
\acp{ODE} are replaced by their so called witness functions, i.e.,
static numerical solutions. We describe the translation of the \acp{ODE}
to their equivalent witness functions in
Section~\ref{sec:converting-odes-into} followed by code generation for a
single \ac{HIOA} through the use of a Mealy \ac{FSM} in
Section~\ref{sec:backendCodeGeneration}.

\subsection{Converting the \acp{ODE} into equivalent witness functions}
\label{sec:converting-odes-into}

Given the \ac{ODE} for $\dot{v_x}$ in location $\mathbf{q_0}$ of the
heart node, \ac{HIOA} defined in Figure~\ref{fig:heartCellHA}
(reproduced in Equation~(\ref{eq:ode})), the witness function computes
the updated value of $v_{x}$ (denoted $v'_{x}$) at discrete time
instants, separated by the so called time step $\delta$
($\delta \in \mathbb{R}^{+}$), using the forward Euler method.  

The witness function, for the \ac{ODE} evolving $v_{x}$, implementing
the forward Euler method is shown in Equation~(\ref{eq:euler_equiv}).
Equation~(\ref{eq:euler_equiv}) evolves $v_{x}$ \emph{iteratively} until
the invariant condition on location $\mathbf{q_{0}}$ ($v < V_{T}$)
holds. The initial value from which $v_{x}$ starts evolving, in
$\mathbf{q_{0}}$ is either: (1) specified by the programmer in the
$Init$ predicate or (2) the final value of $v_{x}$ when an instantaneous
transition is made from location $\mathbf{q_{3}}$ to $\mathbf{q_{0}}$,
shown as update $v^{\prime}_{x} = v_{x}$, shown the edge connecting
these location in Figure~\ref{fig:heartCellHA}.


\begin{equation}
  \dot{v_x} = C_{1} v_x
  \label{eq:ode}
\end{equation}

\begin{equation}
  v^\prime_x = v_x + \delta \times (C_{1} v_x)
  \label{eq:euler_equiv}
\end{equation}

% This iterative evolution of the continuous variables at discrete points
% in time is akin to transitions on a logical \emph{tick} of a synchronous
% program~\cite{benveniste03}. A likeness that we will exploit when
% composing multiple \ac{HIOA} together in Section~\ref{sec:composition}.

\subsection{Backend code generation for a single \ac{HIOA}}
\label{sec:backendCodeGeneration}

\begin{figure}
  \centering \begin{tikzpicture}[->,>=stealth',shorten >=1pt,auto,node distance=4cm,
semithick,scale=0.9, transform shape]
\tikzstyle{every state}=[circle,rounded corners,
minimum height = 1.2cm, text width=1.2cm, text
centered, fill=blue!20,draw=none,text=black,
draw,line width=0.3mm]

\node[state] (Q0)  { $q_0$ };

\path[<-, dashed] (Q0.225) edge node[below, align=left, shift={(0,-0.5)}] {
	\footnotesize initial \\
	\footnotesize $\begin{matrix}
		v_x = 0 \\
		v_y = 0 \\
		v_z = 0 \\
		\theta = 0
	\end{matrix}$
} ++(-1cm,-1cm);

\node[state]
(Q1) [node distance=5cm, right of=Q0] 	 {$q_1$};

\node[state] 
(Q2) [below of=Q1] 	 {$q_2$};

\node[state] 
(Q3) [below of=Q0] 	 { $q_3$};


%inter-location transitions
\path[->] (Q0.20) edge[out=20,in=160] node[align=center] {
  $\frac{\footnotesize \{g(\vec{v_{I}}) > V_T\}}
  {\footnotesize
  \left\{\begin{matrix} v^\prime_x = 0.3v \\ v^\prime_y = 0.0v \\
      v^\prime_z = 0.7v \\ \theta^\prime = v / V_T \end{matrix}\right\}}$
} (Q1.160);

\path[->] (Q1.200) edge[out=200,in=340] node[align=center] {
	$\frac{\footnotesize \{g(\vec{v_{I}}) \leq 0 \wedge v < V_T\}}
	{\footnotesize \left\{\begin{matrix} v^\prime_x = v_x \\ v^\prime_y = v_y \\ v^\prime_z = v_z \end{matrix}\right\}}$
} (Q0.340);

\path[->] (Q1) edge node[align=center] {
	$\frac{\footnotesize \{v \geq V_T\}}
	{\footnotesize \left\{\begin{matrix} v^\prime_x = v_x \\ v^\prime_y = v_y \\ v^\prime_z = v_z \end{matrix}\right\}}$
} (Q2);

\path[->] (Q2) edge node[align=center] {
	$\frac{\footnotesize \{v \geq V_O - 80.1\sqrt{\theta}\}}
	{\footnotesize \left\{\begin{matrix} v^\prime_x = v_x \\ v^\prime_y = v_y \\ v^\prime_z = v_z \end{matrix}\right\}}$
} (Q3);

\path[->] (Q3) edge node[align=center, shift={(1.8,0)}] {
	$\frac{\footnotesize \{v \leq V_R\}}
	{\footnotesize \left\{\begin{matrix} v^\prime_x = v_x \\ v^\prime_y = v_y \\ v^\prime_z = v_z \end{matrix}\right\}}$
} (Q0);

%intra-location transitions
\path[->] (Q0.180) edge[out=180,in=90,distance=1.5cm] node[align=center,
shift={(2.5,0.5)}] {
  $\frac{\footnotesize \{v < V_T \wedge \neg (g(\vec{v_{I}}) \geq V_T)\}}
  {\footnotesize \left\{\begin{matrix}
      v^\prime_x = v_x + \delta C_{1} v_x \\
      v^\prime_y = v_y + \delta C_{2} v_y \\
      v^\prime_z = v_z + \delta C_{3} v_z \\
      v=v^\prime_x - v^\prime_y + v^\prime_z
	\end{matrix}\right\}}$
} (Q0.90);

\path[->] (Q1.90) edge[out=90,in=0,distance=1.5cm] node[align=center,
shift={(-2.5,0.5)}] {
  $\frac{\footnotesize \{v < V_T \wedge \neg (g(\vec{v_{I}}) \leq V_T)\}}
  {\footnotesize \left\{\begin{matrix}
      v^\prime_x = v_x + \delta (C_{4} v_x {+} C_{7} g(\vec{v_{I}})) \\
      v^\prime_y = v_y + \delta (C_{5} v_y {+} C_{8} g(\vec{v_{I}})) \\
      v^\prime_z = v_z + \delta (C_{6} v_z {+} C_{9} g(\vec{v_{I}})) \\
      v=v^\prime_x - v^\prime_y + v^\prime_z
	\end{matrix}\right\}}$
} (Q1.0);

\path[->] (Q2.0) edge[out=0,in=-90,distance=1.5cm] node[align=center,
shift={(-2.5,-0.5)}] {
  $\frac{\footnotesize \{v < V_O - 80.1 \sqrt{\theta}\}}
  {\footnotesize \left\{\begin{matrix}
      v^\prime_x = v_x + \delta C_{10} v_x \\
      v^\prime_y = v_y + \delta C_{11} v_y \\
      v^\prime_z = v_z + \delta C_{12} v_z \\
      v=v^\prime_x - v^\prime_y + v^\prime_z
	\end{matrix}\right\}}$
} (Q2.-90);

\path[->] (Q3.-90) edge[out=-90,in=180,distance=1.5cm]
node[align=center, shift={(2.5,-0.5)}] {
  $\frac{\footnotesize \{v > V_R\}}
  {\footnotesize \left\{\begin{matrix}
      v^\prime_x = v_x + \delta C_{13} v_x \\
      v^\prime_y = v_y + \delta C_{14} v_y \\
      v^\prime_z = v_z + \delta C_{15} v_z \\
      v=v^\prime_x - v^\prime_y + v^\prime_z
	\end{matrix}\right\}}$
} (Q3.180);

\end{tikzpicture}


%%% Local Variables:
%%% mode: latex
%%% TeX-master: "../DATE2016_codegen"
%%% End:

  \caption{\acf{FSM} of a heart node \label{fig:heartCellFSM}. We ignore
    the $\tau$ event in the \ac{FSM}, because it always evaluates to
    $true$.}
\end{figure}

In order to facilitate code generation, each \ac{HIOA} is transformed
into a simple Mealy \ac{FSM}. The \ac{FSM} generated from the \ac{HIOA}
in Figure~\ref{fig:heartCellFSM} is shown in
Figure~\ref{fig:heartCellFSM}.

The translation from a single \ac{HIOA} to its \ac{FSM} representation
is a two step procedure:

\begin{enumerate}
\item \textit{Translating locations into states}: The first step
  translates each location in the \ac{HIOA} into a state in the
  \ac{FSM}. Hence, the set of locations $\{q_{0}, q_{1}, q_{2}, q_{3}\}$
  have an equivalent named set of states in the generated \ac{FSM}.
\item \textit{Translating each edge in the \ac{HIOA} to a transition in
    the \ac{FSM}}: The second step translates each edge ($e$) in the
  \ac{HIOA} into an equivalent transition ($t$) in the \ac{FSM}. Every
  transition ($t$) in the \ac{FSM} is of the form:
  $\frac{guard}{update}$. The $guard$ precedent, is the conjunction of
  the set $\Sigma_{EI} \cup \{\tau\}$ and the conditions specified in
  the $Jump$ predicate on edge $e$. The $update$ antecedent are the
  updates extracted from the $Jump$ predicate on edges. For example, the
  edge $(q_{0}, \tau, \emptyset, q_{1})$ with the $Jump$ predicate as
  shown in Figure~\ref{fig:heartCellHA} translated into a transition in
  the generated \ac{FSM} is
  $\frac{\tau \wedge g(\vec{v_{I}} \geq V_{T})}{v^{\prime}_{x} = 0.3v,
    v^{\prime}_{y}=0.0v, v^{\prime}_{z}=0.7v, \theta^{\prime}=v/V_{T}}$
  also shown in Figure~\ref{fig:heartCellFSM}. Note, that in
  Figure~\ref{fig:heartCellFSM}, we have ignored the $\tau$ events,
  because they always evaluate to $true$.

\end{enumerate}

% Two initial steps are performed: (1) for each location in the
% \ac{HIOA} there is a corresponding state in the \ac{FSM} of the same
% name and (2) for each inter-location edge $e$ in the \ac{HIOA} there
% is a corresponding transition $t$ between the same two states. Each
% transition $t$ has a condition equal to the input events of $e$
% conjuncted with the conditions in $Jump(e)$, and output equal to the
% output events of $e$ combined with the updates present in $Jump(e)$.

The aforementioned steps generate an \ac{FSM} that is able to capture
the instantaneous transitions across locations called
\textit{inter-location} transitions, but it are unable capture the
evolution of the continuous variables via the \acp{ODE}. In order to
replicate the continuous evolution of variables within the locations, we
create \textit{self-transitions} on each state in the \ac{FSM}. A
self-transition by definition has the same source and target state,
e.g., transition from state $q_{0}$ to $q_{0}$ as shown in
Figure~\ref{fig:heartCellFSM}. The invariant in $Inv$ from the initial
\ac{HIOA} becomes the guard of the self-transition, and the witness
functions for each of the \acs{ODE} described in
Section~\ref{sec:converting-odes-into} become the update of the
transition.

Inter-location transitions, for each \ac{HIOA}, in a network of
\ac{HIOA} may be enabled by external inputs (continuous variables from
set $X_{EI}$ or discrete events from set $\Sigma_{EI}$). For example,
the edge connecting $q_{0}$ to $q_{1}$, in Figure~\ref{fig:heartCellHA},
is enabled as soon as the aggregate voltage of neighbouring node
($g(\vec{v_{I}})$) is greater than or equal to some thresh hold voltage
$V_{T}$. In order to ensure that such inter-location transitions are
enabled correctly, we need to enforce that all self-transitions in the
generated \ac{FSM} are the lowest priority transitions. This is
accomplished by updating the guard condition of the self-transitions as
follows:
\begin{itemize}
\item
  $guard \leftarrow guard \wedge \bigwedge_{\forall \sigma \in
    \Sigma_{EI}} \neg \sigma$
\end{itemize}

In the \ac{HIOA} of Figure~\ref{fig:heartCellHA},
$\Sigma_{EI} = \emptyset$, hence, the self-transition guard conditions
remain the same.

% to be conjuncted with the negation of the disjunction of all conditions
% of egress transitions from the state, i.e.
% $Cond(self) \wedge \neg (Cond(trans_{0}) \vee \dots \vee
% Cond(trans_{n})$.
% For location $\mathbf{q_0}$ in the \ac{HIOA} the invariant $v < V_{T}$
% will become a self transition on state $\mathbf{q_0}$ in the \ac{FSM}
% with a condition of $v < V_{T} \wedge \neg (g(\vec{v_{I}}) \geq V_{T})$.
% The final corresponding \ac{FSM} for the \ac{HIOA} from
% Figure~\ref{fig:heartCellHA} is shown in Figure~\ref{fig:heartCellFSM}.

Each of the generated \ac{FSM}, in the network, may be individually
transformed into a switch case statement in `C'-code, with a
corresponding \texttt{Initialization} and \texttt{Input-Output}
functions to initialize the \ac{FSM} and communicate with its
environment (nodes in case of the network of \ac{HIOA} for the heart),
respectively.

% The \acp{FSM} is then transformed into C code which contains an
% \emph{Initialisation Function} corresponding to the $Init$ of the
% \ac{HIOA} as well as the setting of the initial state, and a \emph{Run
%   Function} which, given an existing state $l$ and values for all
% variables $X$ and events $\Sigma$, performs a single transition and
% updates any variables or events that may have changed.

%%% Local Variables:
%%% mode: latex
%%% TeX-master: "../DATE2016_codegen"
%%% End:
