\section{Modular code generation}
\label{sec:codeGen}

In this section we present our approach for modular code generation from
a network of \acp{HIOA} as it is implemented in our tool, called
\ourTool.  An overview of the solution is presented in
Figure~\ref{fig:overview}, with the first of the two steps being
outlined previously in Section~\ref{sec:SHA}, and the second step in
Section~\ref{sec:backendCodeGeneration} below.


\subsection{Backend code generation}
\label{sec:backendCodeGeneration}

Here we describe step $2$ of the compilation process from
Figure~\ref{fig:overview}. Once a network of \acp{SHIOA} has been
created, code needs to be generated to implement the functionality of
each \ac{SHIOA}.

\begin{figure}
  \centering
  \begin{tikzpicture}[->,>=stealth',shorten >=1pt,auto,node distance=4cm,
semithick,scale=0.9, transform shape]
\tikzstyle{every state}=[circle,rounded corners,
minimum height = 1.2cm, text width=1.2cm, text
centered, fill=blue!20,draw=none,text=black,
draw,line width=0.3mm]

\node[state] (Q0)  { $q_0$ };

\path[<-, dashed] (Q0.225) edge node[below, align=left, shift={(0,-0.5)}] {
	\footnotesize initial \\
	\footnotesize $\begin{matrix}
		v_x = 0 \\
		v_y = 0 \\
		v_z = 0 \\
		\theta = 0
	\end{matrix}$
} ++(-1cm,-1cm);

\node[state]
(Q1) [node distance=5cm, right of=Q0] 	 {$q_1$};

\node[state] 
(Q2) [below of=Q1] 	 {$q_2$};

\node[state] 
(Q3) [below of=Q0] 	 { $q_3$};


%inter-location transitions
\path[->] (Q0.20) edge[out=20,in=160] node[align=center] {
  $\frac{\footnotesize \{g(\vec{v_{I}}) > V_T\}}
  {\footnotesize
  \left\{\begin{matrix} v^\prime_x = 0.3v \\ v^\prime_y = 0.0v \\
      v^\prime_z = 0.7v \\ \theta^\prime = v / V_T \end{matrix}\right\}}$
} (Q1.160);

\path[->] (Q1.200) edge[out=200,in=340] node[align=center] {
	$\frac{\footnotesize \{g(\vec{v_{I}}) \leq 0 \wedge v < V_T\}}
	{\footnotesize \left\{\begin{matrix} v^\prime_x = v_x \\ v^\prime_y = v_y \\ v^\prime_z = v_z \end{matrix}\right\}}$
} (Q0.340);

\path[->] (Q1) edge node[align=center] {
	$\frac{\footnotesize \{v \geq V_T\}}
	{\footnotesize \left\{\begin{matrix} v^\prime_x = v_x \\ v^\prime_y = v_y \\ v^\prime_z = v_z \end{matrix}\right\}}$
} (Q2);

\path[->] (Q2) edge node[align=center] {
	$\frac{\footnotesize \{v \geq V_O - 80.1\sqrt{\theta}\}}
	{\footnotesize \left\{\begin{matrix} v^\prime_x = v_x \\ v^\prime_y = v_y \\ v^\prime_z = v_z \end{matrix}\right\}}$
} (Q3);

\path[->] (Q3) edge node[align=center, shift={(1.8,0)}] {
	$\frac{\footnotesize \{v \leq V_R\}}
	{\footnotesize \left\{\begin{matrix} v^\prime_x = v_x \\ v^\prime_y = v_y \\ v^\prime_z = v_z \end{matrix}\right\}}$
} (Q0);

%intra-location transitions
\path[->] (Q0.180) edge[out=180,in=90,distance=1.5cm] node[align=center,
shift={(2.5,0.5)}] {
  $\frac{\footnotesize \{v < V_T \wedge \neg (g(\vec{v_{I}}) \geq V_T)\}}
  {\footnotesize \left\{\begin{matrix}
      v^\prime_x = v_x + \delta C_{1} v_x \\
      v^\prime_y = v_y + \delta C_{2} v_y \\
      v^\prime_z = v_z + \delta C_{3} v_z \\
      v=v^\prime_x - v^\prime_y + v^\prime_z
	\end{matrix}\right\}}$
} (Q0.90);

\path[->] (Q1.90) edge[out=90,in=0,distance=1.5cm] node[align=center,
shift={(-2.5,0.5)}] {
  $\frac{\footnotesize \{v < V_T \wedge \neg (g(\vec{v_{I}}) \leq V_T)\}}
  {\footnotesize \left\{\begin{matrix}
      v^\prime_x = v_x + \delta (C_{4} v_x {+} C_{7} g(\vec{v_{I}})) \\
      v^\prime_y = v_y + \delta (C_{5} v_y {+} C_{8} g(\vec{v_{I}})) \\
      v^\prime_z = v_z + \delta (C_{6} v_z {+} C_{9} g(\vec{v_{I}})) \\
      v=v^\prime_x - v^\prime_y + v^\prime_z
	\end{matrix}\right\}}$
} (Q1.0);

\path[->] (Q2.0) edge[out=0,in=-90,distance=1.5cm] node[align=center,
shift={(-2.5,-0.5)}] {
  $\frac{\footnotesize \{v < V_O - 80.1 \sqrt{\theta}\}}
  {\footnotesize \left\{\begin{matrix}
      v^\prime_x = v_x + \delta C_{10} v_x \\
      v^\prime_y = v_y + \delta C_{11} v_y \\
      v^\prime_z = v_z + \delta C_{12} v_z \\
      v=v^\prime_x - v^\prime_y + v^\prime_z
	\end{matrix}\right\}}$
} (Q2.-90);

\path[->] (Q3.-90) edge[out=-90,in=180,distance=1.5cm]
node[align=center, shift={(2.5,-0.5)}] {
  $\frac{\footnotesize \{v > V_R\}}
  {\footnotesize \left\{\begin{matrix}
      v^\prime_x = v_x + \delta C_{13} v_x \\
      v^\prime_y = v_y + \delta C_{14} v_y \\
      v^\prime_z = v_z + \delta C_{15} v_z \\
      v=v^\prime_x - v^\prime_y + v^\prime_z
	\end{matrix}\right\}}$
} (Q3.180);

\end{tikzpicture}


%%% Local Variables:
%%% mode: latex
%%% TeX-master: "../DATE2016_codegen"
%%% End:

  \caption{\acf{FSM} of a heart cell \label{fig:heartCellFSM}. We ignore
    the $\tau$ event in the FSM, because it always evaluates to $true$.}
\end{figure}

In order to facilitate code generation, each \ac{SHIOA} is transformed
into a simple Mealy \ac{FSM} (such as that shown in
Figure~\ref{fig:heartCellFSM}) where each step of the program requires a
transition be taken.  This process involves creating self loops on each
state in $Loc$, where the invariant in $Inv$ becomes the guard of the
transition, and the \acs{ODE} in $Switness$ become the outputs of the
transition.  The corresponding \ac{FSM} for the \ac{SHIOA} from
Figure~\ref{fig:heartCellSHA} is shown in Figure~\ref{fig:heartCellFSM}.

Each of these \acp{FSM} are then transformed into C code which contains
an \emph{Initialisation Function} corresponding to the $Init$ of the
\ac{FSM}, and a \emph{Run Function} which, given an existing state,
performs a single transition and updates any signals that may have
changed.


\subsection{Parallel Composition}
\label{sec:composition}

\begin{figure}
  \centering
  \begin{tikzpicture}[->,>=stealth',shorten >=1pt,auto,node distance=4cm,
semithick,scale=0.9, transform shape]
\tikzstyle{every state}=[circle,rounded corners,
minimum height = 1.2cm, text width=1.2cm, text
centered, fill=blue!20,draw=none,text=black,
draw,line width=0.3mm]

\node[state] (IO)  { $I/O$ };

\path[<-, dashed] (IO.90) edge node[above, align=center, shift={(0.6,0)}] {
	initial
} ++(0cm,1cm);

\node[state]
(Cell1) [below right=0.37cm and 0.9cm of IO] 	 {$Cell1$};

\node[state]
(Cell2) [below left=1.28cm and -0.33cm of Cell1] 	 {$Cell2$};

\node[state]
(Cell33) [below left=0.37cm and 0.9cm of IO] 	 {$Cell33$};

\node[state]
(Cell32) [below right=1.28cm and -0.33cm of Cell33] 	 {$Cell32$};


\path[->] (IO.0) edge[out=0,in=108] (Cell1.108);

\path[->] (Cell1.-72) edge[out=-72,in=36] (Cell2.36);

\path[->, dotted] (Cell2.-144) edge[out=-144,in=-36] (Cell32.-36);

\path[->] (Cell32.-216) edge[out=-216,in=-108] (Cell33.-108);

\path[->] (Cell33.-288) edge[out=-288,in=-180] (IO.-180);

\end{tikzpicture}
  \caption{Synchronous composition of multiple heart
    cells \label{fig:heartCellComposition}}
\end{figure}

In order to compose each of the \acp{FSM} together, we take inspiration
from the concepts of Synchronous Languages such as SL~\cite{SlLanguage}.
The concept of Ticks and Reactions are carried over, whereby each
\ac{FSM} performs only a single step (``tick'') until all \acp{FSM} have
completed a step (``reaction'') and the process can repeat.  In order to
deal with data dependencies, we also implement the concept of \emph{pre}
whereby the value of all inputs to each \ac{FSM} are not updated with
new values until the end of that reaction.  The concept of \emph{pre}
also enables \ourTool to simplify the process of handling cyclic
\acp{ODE} (an issue that other tools do not
consider~\cite{kim2003modular}) by causing the dependencies to refer to
previous, rather than current, values.

The implementation of this in \ourTool is done by creating a round-robin
scheduler which executes one tick of each \ac{FSM} in series, followed
by I/O synchronisation at the end of each reaction.  Such I/O
synchronisation deals with the emission of all outputs from the system,
the intake of all inputs to the system, and the transfer of signals
between \acp{FSM} in the network.  This idea is illustrated in
Figure~\ref{fig:heartCellComposition}.  This concept is amenable to
further scheduling algorithms that need not be sequential in nature.
\ourTool can be extended to support parallel computation by executing
different \acp{FSM} on separate threads, with synchronisation between
threads occurring at the end of each reaction.


%%% Local Variables:
%%% mode: latex
%%% TeX-master: "../DATE2016_codegen"
%%% End:
