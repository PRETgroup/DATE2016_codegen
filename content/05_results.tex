\section{Benchmarking}
\label{sec:benchmarking}


We present a set of experiments to evaluate the efficacy of the proposed modular code generation tool (\ourTool) with Simulink\textsuperscript{\textregistered}. 
In the first experiment, we evaluate the \emph{scalability} of \ourTool as the number of cells in the heart model increases. 
In the second experiment, we select benchmarks that span across different application domains such as medical, physics, and industrial automation, to illustrate the \emph{diversity} of the proposed approach.
We then compare each of these benchmarks against Simulink\textsuperscript{\textregistered} in terms of execution time and code size.


\subsection{Scalability}

\begin{figure}[htbp]
	\centering
	\begin{tikzpicture}
\begin{axis}
[ xlabel={Number of Heart Cells},
ylabel={Execution Time ({s})},
axis y line = left,
axis x line = bottom,
xmin=0000,   xmax=1000,
ymin=0,
extra tick style={grid=major}
]
\addplot[color=blue!90,
mark=.,
mark size=2,
smooth
]
table [x=n, y=t, col sep=comma] {./figures/scalabilityGraphData.csv};
\end{axis}
\end{tikzpicture}
	\caption{Scalability of \ourTool's execution time against number of cells}
	\label{fig:scalability}
\end{figure}

It scales!

\subsection{Diversity}

\begin{figure}[htbp]
	\centering
	\begin{tikzpicture}
\begin{axis}[
	ybar,
	enlargelimits=0.15,
	legend style={
		at={(0.5,-0.15)},
		anchor=north,
		legend columns=-1
	},
	ylabel={Execution time (ms)},
	symbolic x coords={TSN, NHC, WH, MTG, NP},
	xtick=data,
	nodes near coords,
	nodes near coords align={vertical},
]

%Simulink
\addplot coordinates {
	(TSN,0)
	(NHC,14254)
	(WH,0)
	(MTG,0)
	(NP,0)
};

%Piha (O0)
\addplot coordinates {
	(TSN,316)
	(NHC,1997)
	(WH,0)
	(MTG,249)
	(NP,504)
};

%Piha (O2)
\addplot coordinates {
	(TSN,119)
	(NHC,713)
	(WH,0)
	(MTG,113)
	(NP,207)
};

\legend{\simulink, \ourTool (O0), \ourTool (O2)}

\end{axis}
\end{tikzpicture}

	\caption{Comparison of the execution time (in ms) between Simulink\textsuperscript{\textregistered} and \ourTool for the benchmarks in Table~\ref{tab:benchmarks}.}
	\label{fig:executionTime}
\end{figure}

For the purposes of this experiment, we use the five benchmarks presented in Table~\ref{tab:benchmarks}.
The table also presents the number of locations (\#L) in each hybrid automata.
For example, $(2, 2, 2)$ denotes that the \acf{TTS} benchmark is described using three \acp{HA} each with two locations.
More details about the benchmarks and their implementation in \ourTool and Simulink\textsuperscript{\textregistered} are available online~\cite{githubBenchmarks}.

\begin{table*}
	\centering
	\caption{Benchmark descriptions
	\label{tab:benchmarks}}
\begin{tabular}{ | c | c | c | l | } \hline
\textbf{Benchmarks}
	& \textbf{Domain} 
	& \textbf{\#L } 
	& \textbf{Description} \\ \hline

	\acf{TTS}
		& Physics~\cite{Pedro2005}
		& $(2, 2, 2)$
		& Three thermostats heating a room to keep it warm\\ \hline
		
	\acf{NHC}
		& Biology~\cite{chen201487}
		& $(4, 4, ..., 4)$
		& Captures the electrical conduction system of a heart with $33$ nodes\\ \hline

	\acf{WH}
		& Physics~\cite{raskin05}
		& $(4, 4)$
		& Models the heating of water in a tank \\ \hline
		
	\acf{MTG}  
		& Industrial automation~\cite{Costello2013}
		& $(2, 3, ..., 3)$
		& Models the behaviour of a gate and $30$ trains at a rail road crossing\\ \hline
		
	\acf{NP}
		& Industrial automation~\cite{alur2015book}
		& $(3, 3)$
		& Switches between two fuel rods to avoid nuclear meltdown\\ \hline
	
	
 \end{tabular}
 \end{table*}