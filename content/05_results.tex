\section{Benchmarking}
\label{sec:benchmarking}


We present a set of experiments to 
evaluate  the efficacy of the proposed 
 modular code generation tool (\ourTool) 
with Simulink. 
In the first experiment, we evaluate the \emph{scalability} 
of \ourTool as the number of cells in the heart model
increase exponentially. 
In the second experiment, we select benchmarks 
that span across different application
domains such as medical, physics, and 
industrial automation, to illustrating the \emph{diversity} of the proposed 
approach. 
In both experiments we compare against Simulink in terms of compilation time,
execution time and code size.


\subsection{Scalability}
\begin{figure}[htbp]
	\centering
	\begin{tikzpicture}[yscale=0.85]
\begin{axis}
[
	xlabel={Number of Heart Nodes},
	ylabel={Execution Time ({s})},
	axis y line = left,
	axis x line = bottom,
	xmin=0000,
	xmax=1000,
	ymin=0,
	ymax=150,
	ytick={0, 10, 50, 100, 150},
	yticklabels={0, 10, 50, 100, 150},
	extra tick style={grid=major},
	legend style={
		at={(0.52,0.99)},
		anchor=north,
		legend columns=-1
	}
]

\addplot[color=blue!90,
	mark=square*,
	mark size=2
] table [
	x=n,
	y=s_t,
	col sep=comma
] {./figures/scalabilityGraphData.csv};
\addlegendentry{\simulink}

\addplot[color=red!90,
	mark=*,
	mark size=2
] table [
	x=n,
	y=p_t,
	col sep=comma
] {./figures/scalabilityGraphData.csv};
\addlegendentry{\ourTool}

\addplot[color=black!90,
	dashed,
	mark=.,
	mark size=2
] coordinates {
	(0,10)
	(1000,10)
} node[pos=0.8] (endofrealtime) {};
\node[above] at (endofrealtime) {Real Time};

\addplot[color=blue!90,
	dashed,
	mark=.,
	mark size=2
] coordinates {
	(297,126.448)
	(297,0)
};



\end{axis}
\end{tikzpicture}
	\caption{Scalability of Piha's execution time against number of cells}
	\label{fig:scalability}
\end{figure}

\subsection{Diversity}
For the purposes of this experiment, we use the five benchmarks presented in
Table~\ref{tab:benchmarks}. 


As depicted in column one of Table~\ref{tab:benchmarks}, four out of the 
seven benchmarks are described using a single \ac{HA} and 
the remaining three examples are described using more than one \ac{HA}.
The table also presents the 
number of locations (\#L) in each hybrid automata.
For example, (2,3) denotes that the Train Gate Control
(TG) benchmark is described using a HA with two locations 
and a second HA with three locations. More details
about the benchmarks and their 
implementation in 
%\ourTool 
and Simulink
are available online~\cite{githubBenchmarks}.

\begin{table*}
	\centering
	\caption{Benchmark descriptions
	\label{tab:benchmarks}}
\begin{tabular}{ | c | c | c | l | } \hline
\textbf{Benchmarks}
	& \textbf{Domain} 
	& \textbf{\#L } 
	& \textbf{Description} \\ \hline

	\acf{TTS}
		& Physics~\cite{Pedro2005}
		& $(2, 2, 2)$
		& Three thermostats heating a room to keep it warm\\ \hline
		
	\acf{NHC}
		& Biology~\cite{chen201487}
		& $(4^{33})$
		& Captures the electrical conduction system of a heart with $33$ nodes\\ \hline

	\acf{WH}
		& Physics~\cite{raskin05}
		& $(4, 4)$
		& Models the heating of water in a tank \\ \hline
		
	\acf{MTG}  
		& Industrial automation~\cite{Costello2013}
		& $(2, 3^{30})$
		& Models the behaviour of a gate and $30$ trains at a rail road crossing\\ \hline
		
	\acf{NP}
		& Industrial automation~\cite{alur2015book}
		& $(3^{30}, 3^{30})$
		& Switches between two fuel rods in 30 reactors to avoid nuclear meltdown\\ \hline
	
	
 \end{tabular}
 \end{table*}