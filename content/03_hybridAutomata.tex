\section{Hybrid Automata}
\label{sec:HA}



\acf{HA} is ideal for describing 
Cyber-physical systems. As explained in the Introduction,
non-deterministic of \ac{HA} are not suitable for modelling and
emulation of a plant. To address this problem, based on the
synchronous approach~\cite{benveniste03}, we propose a deterministic semantics for \ac{HA}.
In this section, we (1) present an example \ac{HA} 
followed by a formal definition.
Then, we present deterministic semantics for \ac{HA} 
along with a proof for determinism.



\subsection{Example of \ac{HA} (TEXT NEEDS TO BE UPDATED)}

\begin{figure}
\centering

\begin{tikzpicture}[->,>=stealth',shorten >=1pt,auto,
node distance=4cm,
semithick,scale=0.8, transform shape]
\tikzstyle{every state}=[rectangle,rounded corners,
 minimum height = 1.9cm, text width=3.35cm, text centered, fill=blue!20,draw=none,text=black, draw,line width=0.3mm]
 \tikzset{
 	atrialCell/.style={
 		fill=green!20,
 		circle,
 		minimum size=1.5cm,
 		draw,
 		line width=0.2mm
 	}
 }
 


\node[state, 
label={[shift={(0,0)}]$v < V_T$}, 
label={[shift={(-1.5,-2.5)}]\large $ \mathbf{q_0}$ }]
(Q0)  {$\dot{v_x}=C_1 v_x$ \\ $\dot{v_y}=C_2 v_y$ \\ $\dot{v_z}=C_3 v_z$ \\ $v=v_x - v_y + v_z$};

%entry
\path[<-, dashed] (Q0.150) edge node[below, align=left, shift={(1.15,1)}] {
	\footnotesize initial \\
	\footnotesize $\begin{matrix}
		v_{x}=0 \wedge v_{y}=0 \\
		\quad \wedge v_{z}=0 \wedge \theta=0
	\end{matrix}$
} ++(0cm,1.25cm);




\node[state, 
label={[shift={(0,0)}]$v < V_T$}, 
label={[shift={(-1.5,-2.5)}]\large $\mathbf{q_1}$   }]
(Q1) [node distance=6.3cm, right of=Q0] {$\dot{v_x}=C_4 v_x + C_7 g(\vec{v_{I}})$ \\ $\dot{v_y}=C_5 v_y + C_8 g(\vec{v_{I}})$ \\ $\dot{v_z}=C_6 v_z + C_9 g(\vec{v_{I}})$ \\ $v=v_x - v_y + v_z$};

\node[state, 
label={[shift={(0.2,-2.5)}]$v < V_O - 80.1 \sqrt{\theta}$}, 
label={[shift={(-1.5,-2.5)}]\large $\mathbf{q_2}$  }] 
(Q2) [below of=Q1] {$\dot{v_x}=C_{10} v_x$ \\ $\dot{v_y}=C_{11} v_y$ \\ $\dot{v_z}=C_{12} v_z$ \\ $v=v_x - v_y + v_z$};

\node[state, 
label={[shift={(0,-2.5)}]$v > V_R$}, 
label={[shift={(-1.5,-2.5)}]\large $\mathbf{q_3}$  }]
(Q3) [below of=Q0] {$\dot{v_x}=C_{13} v_x f(\theta)$ \\ $\dot{v_y}=C_{14} v_y f(\theta)$ \\ $\dot{v_z}=C_{15} v_z$ \\ $v=v_x - v_y + v_z$};

\path[->] (Q0.10) edge node[align=center] {
	\footnotesize $\{\tau\}$, \\
	\footnotesize $\{g(\vec{v_{I}}) \geq V_T\}$, \\
	\footnotesize $\left\{\begin{matrix} v^\prime_x = 0.3v \\ v^\prime_y = 0.0v \\ v^\prime_z = 0.7v \\ \theta^\prime = v / V_T \end{matrix}\right\}$
} (Q1.170);

\path[->] (Q1.190) edge node[align=center] {
	\footnotesize $\{\tau\}$, \\
	\footnotesize $\{g(\vec{v_{I}}) \leq 0 \wedge v < V_T$\}, \\
	\footnotesize $\left\{\begin{matrix} v^\prime_x = v_x \\ v^\prime_y = v_y \\ v^\prime_z = v_z \end{matrix}\right\}$
} (Q0.350);

\path[->] (Q1) edge node[align=center] {
	\footnotesize $\{\tau\}$, \\
	\footnotesize $\{v \geq V_T\}$, \\
	\footnotesize $\left\{\begin{matrix} v^\prime_x = v_x \\ v^\prime_y = v_y \\ v^\prime_z = v_z \end{matrix}\right\}$
} (Q2);

\path[->] (Q2) edge node[align=center] {
	\footnotesize $\{\tau\}$, \\
	\footnotesize $\{v \geq V_O - 80.1\sqrt{\theta}\}$, \\
	\footnotesize $\left\{\begin{matrix} v^\prime_x = v_x \\ v^\prime_y = v_y \\ v^\prime_z = v_z \end{matrix}\right\}$
} (Q3);

\path[->] (Q3) edge node[align=center, shift={(1.8,0)}] {
	\footnotesize $\{\tau\}$, \\
	\footnotesize $\{v \leq V_R\}$, \\
	\footnotesize $\left\{\begin{matrix} v^\prime_x = v_x \\ v^\prime_y = v_y \\ v^\prime_z = v_z \end{matrix}\right\}$
} (Q0);


\node[ draw,   inner xsep=1.1cm, inner ysep=1.9cm, shift={(0, 0.35)},
label={[label distance=0cm]60:~}, 
fit= (Q0)(Q2)] (BOX) {};

%--------------------------- nodes on top ------------
\node[atrialCell, above of = Q0, xshift=2.5cm, node distance = 3.9cm,
label={[shift={(1,-0.4)}]\large $v$ },
label={[shift={(1.1,-1.2)}]\large $v_{I_{0}}$ }]
(AN1){$N_1$};
\node[atrialCell, right of = AN1, node distance = 4cm,
label={[shift={(-1.1,-0.4)}]\large $v_{I_{0}}$ },
label={[shift={(-1,-1.2)}]\large $v$ }] 
(AN2){$N_2$};
\draw[->](AN1.30) -- (AN2.150);
\draw[->](AN2.210) -- (AN1.-30);


\draw[-,dashed](AN2.-45) -- (BOX.north east);
\draw[-,dashed](AN2.230) -- (BOX.north west);

%---------------- pacemaker I/O ------------
\node[ draw,  rounded corners=2mm, inner xsep=0.7cm,inner ysep=0.4cm,
label={[label distance=0cm]60:~}, 
fit= (AN1)(AN2)] (BOX2) {};

\draw[<-, thick](BOX2.170) -- node[above, shift={(-0.8,0.1)}] {\footnotesize input stimulus ($v_p$)} ++(-1cm,0cm);

\draw[->, thick](BOX2.190) -- node[above, shift={(-0.8,0.1)}] {\footnotesize output trace ($v_s$)} ++(-1cm,0cm);


\end{tikzpicture}


%%% Local Variables:
%%% mode: latex
%%% TeX-master: "../DATE2016_codegen"
%%% End:

\caption{\acf{HA} of a heart cell}
\end{figure}

An HA~\cite{lynch03} is defined using Definition~\ref{def:ha}. For the
water tank example shown in Figure~\ref{fig:waterTankHAtank},
$Loc=\{t_1, t_2, t_3, t_4\}$, $\Sigma=\{ON, OFF, \tau\}$, and an example
edge in $Edge$ is $(t_1, ON, t_2)$. There is a single continuous
variable $x$ representing the temperature of the water in the
tank. Hence, $X=\{x\}$, $\dot{X}=\{\dot{x}\}$, and $X'=\{x'\}$. The only
location that is marked initial is $t_1$ and $Init(t_1): x=20$. An
example flow predicate is $Flow(t_1): \dot{x}=0$, which specifies that
the temperature of the water inside the tank does not change.  A jump
predicate assigned to an edge specifies the condition needed to take a
transition and the updating of values of some continuous variables when
the transition is taken.  For example,
$Jump(t_4, \tau, t_1): x=20 \wedge x'=x$ specifies that the transition
is taken when the value of the temperature is $20^\circ$C and the
updated value of the temperature is also $20^\circ$C. We now formalise
the HA using Definition~\ref{def:ha}.

\subsection{Definition of \ac{HA}}

\begin{definition}
	A hybrid automata is \newline 
	$H = \langle Loc, Edge, \Sigma, Inv, Flow, Jump \rangle$ where
	\begin{itemize}
		\item $Loc=\{l_1,..,l_n\}$ representing $n$ control modes or
		locations.
		\item $Edge \subseteq Loc \times \Sigma \times Loc$ are the set of
		edges between locations.
		\item $\Sigma=\Sigma_I \cup \Sigma_O \cup \{\iSignal\}$ is set of
		event names comprising of input, output, and internal events.
		\item Three sets for the set of continuous variables, their rate of
		change and their updated values represented as follows: \newline
		$X=\{x_1,.., x_m\}$, $\dot{X}=\{\dot{x_1},.., \dot{x_m}\}$, and \newline
		$X'=\{x_{1}',.., x_{m}'\}$.
		\item $Init(l)$: Is a predicate whose free variables are from $X$. It
		specifies the possible valuations of these when the HA starts in
		$l$.
		\item $Inv(l)$: Is a predicate whose free variables are from $X$ and
		it constrains these when the HA resides in $l$.
		\item $Flow(l)$: Is a predicate whose free variables are from
		$X \cup \dot{X}$ and it specifies the rate of change of these
		variables when the HA resides in $l$.
		\item $Jump(e)$: Is a function that assigns to the edge `$e$' a
		predicate whose free variables are from $X \cup X'$. This predicate
		specifies when the mode switch using `$e$' is possible. It also
		specifies the updated values of the variables when this mode switch
		happens.
	\end{itemize}
	\label{def:ha}
\end{definition}


\subsection{Deterministic semantics of \ac{HA}}

\subsection{Proof for determinism}