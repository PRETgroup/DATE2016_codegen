\section{Hybrid Automata}
\label{sec:HA}

\acf{HIOA} is ideal for describing \ac{CPS}. It is very expressive,
non-deterministic and has been successfully used for capturing the
behaviour of a plant in control systems. In this section, we use the
\ac{HIOA} in Figure~\ref{fig:heartCellHA}, which captures the \ac{AP} in
Figure~\ref{fig:actionPotential} as a running example to provide a
formal definition and an informal description of the semantics of
\ac{HIOA}.

\subsection{Example of \acf{HIOA} }

\begin{figure}
  \centering 
\begin{tikzpicture}[->,>=stealth',shorten >=1pt,auto,
node distance=4cm,
semithick,scale=0.8, transform shape]
\tikzstyle{every state}=[rectangle,rounded corners,
 minimum height = 1.9cm, text width=3.35cm, text centered, fill=blue!20,draw=none,text=black, draw,line width=0.3mm]
 \tikzset{
 	atrialCell/.style={
 		fill=green!20,
 		circle,
 		minimum size=1.5cm,
 		draw,
 		line width=0.2mm
 	}
 }
 


\node[state, 
label={[shift={(0,0)}]$v < V_T$}, 
label={[shift={(-1.5,-2.5)}]\large $ \mathbf{q_0}$ }]
(Q0)  {$\dot{v_x}=C_1 v_x$ \\ $\dot{v_y}=C_2 v_y$ \\ $\dot{v_z}=C_3 v_z$ \\ $v=v_x - v_y + v_z$};

%entry
\path[<-, dashed] (Q0.150) edge node[below, align=left, shift={(1.15,1)}] {
	\footnotesize initial \\
	\footnotesize $\begin{matrix}
		v_{x}=0 \wedge v_{y}=0 \\
		\quad \wedge v_{z}=0 \wedge \theta=0
	\end{matrix}$
} ++(0cm,1.25cm);




\node[state, 
label={[shift={(0,0)}]$v < V_T$}, 
label={[shift={(-1.5,-2.5)}]\large $\mathbf{q_1}$   }]
(Q1) [node distance=6.3cm, right of=Q0] {$\dot{v_x}=C_4 v_x + C_7 g(\vec{v_{I}})$ \\ $\dot{v_y}=C_5 v_y + C_8 g(\vec{v_{I}})$ \\ $\dot{v_z}=C_6 v_z + C_9 g(\vec{v_{I}})$ \\ $v=v_x - v_y + v_z$};

\node[state, 
label={[shift={(0.2,-2.5)}]$v < V_O - 80.1 \sqrt{\theta}$}, 
label={[shift={(-1.5,-2.5)}]\large $\mathbf{q_2}$  }] 
(Q2) [below of=Q1] {$\dot{v_x}=C_{10} v_x$ \\ $\dot{v_y}=C_{11} v_y$ \\ $\dot{v_z}=C_{12} v_z$ \\ $v=v_x - v_y + v_z$};

\node[state, 
label={[shift={(0,-2.5)}]$v > V_R$}, 
label={[shift={(-1.5,-2.5)}]\large $\mathbf{q_3}$  }]
(Q3) [below of=Q0] {$\dot{v_x}=C_{13} v_x f(\theta)$ \\ $\dot{v_y}=C_{14} v_y f(\theta)$ \\ $\dot{v_z}=C_{15} v_z$ \\ $v=v_x - v_y + v_z$};

\path[->] (Q0.10) edge node[align=center] {
	\footnotesize $\{\tau\}$, \\
	\footnotesize $\{g(\vec{v_{I}}) \geq V_T\}$, \\
	\footnotesize $\left\{\begin{matrix} v^\prime_x = 0.3v \\ v^\prime_y = 0.0v \\ v^\prime_z = 0.7v \\ \theta^\prime = v / V_T \end{matrix}\right\}$
} (Q1.170);

\path[->] (Q1.190) edge node[align=center] {
	\footnotesize $\{\tau\}$, \\
	\footnotesize $\{g(\vec{v_{I}}) \leq 0 \wedge v < V_T$\}, \\
	\footnotesize $\left\{\begin{matrix} v^\prime_x = v_x \\ v^\prime_y = v_y \\ v^\prime_z = v_z \end{matrix}\right\}$
} (Q0.350);

\path[->] (Q1) edge node[align=center] {
	\footnotesize $\{\tau\}$, \\
	\footnotesize $\{v \geq V_T\}$, \\
	\footnotesize $\left\{\begin{matrix} v^\prime_x = v_x \\ v^\prime_y = v_y \\ v^\prime_z = v_z \end{matrix}\right\}$
} (Q2);

\path[->] (Q2) edge node[align=center] {
	\footnotesize $\{\tau\}$, \\
	\footnotesize $\{v \geq V_O - 80.1\sqrt{\theta}\}$, \\
	\footnotesize $\left\{\begin{matrix} v^\prime_x = v_x \\ v^\prime_y = v_y \\ v^\prime_z = v_z \end{matrix}\right\}$
} (Q3);

\path[->] (Q3) edge node[align=center, shift={(1.8,0)}] {
	\footnotesize $\{\tau\}$, \\
	\footnotesize $\{v \leq V_R\}$, \\
	\footnotesize $\left\{\begin{matrix} v^\prime_x = v_x \\ v^\prime_y = v_y \\ v^\prime_z = v_z \end{matrix}\right\}$
} (Q0);


\node[ draw,   inner xsep=1.1cm, inner ysep=1.9cm, shift={(0, 0.35)},
label={[label distance=0cm]60:~}, 
fit= (Q0)(Q2)] (BOX) {};

%--------------------------- nodes on top ------------
\node[atrialCell, above of = Q0, xshift=2.5cm, node distance = 3.9cm,
label={[shift={(1,-0.4)}]\large $v$ },
label={[shift={(1.1,-1.2)}]\large $v_{I_{0}}$ }]
(AN1){$N_1$};
\node[atrialCell, right of = AN1, node distance = 4cm,
label={[shift={(-1.1,-0.4)}]\large $v_{I_{0}}$ },
label={[shift={(-1,-1.2)}]\large $v$ }] 
(AN2){$N_2$};
\draw[->](AN1.30) -- (AN2.150);
\draw[->](AN2.210) -- (AN1.-30);


\draw[-,dashed](AN2.-45) -- (BOX.north east);
\draw[-,dashed](AN2.230) -- (BOX.north west);

%---------------- pacemaker I/O ------------
\node[ draw,  rounded corners=2mm, inner xsep=0.7cm,inner ysep=0.4cm,
label={[label distance=0cm]60:~}, 
fit= (AN1)(AN2)] (BOX2) {};

\draw[<-, thick](BOX2.170) -- node[above, shift={(-0.8,0.1)}] {\footnotesize input stimulus ($v_p$)} ++(-1cm,0cm);

\draw[->, thick](BOX2.190) -- node[above, shift={(-0.8,0.1)}] {\footnotesize output trace ($v_s$)} ++(-1cm,0cm);


\end{tikzpicture}


%%% Local Variables:
%%% mode: latex
%%% TeX-master: "../DATE2016_codegen"
%%% End:

  \caption{Top of the figure shows the interaction between two heart nodes 
    ($N_1$ and $N_2$). Each node is defined as a \acf{HIOA} 
    \label{fig:heartCellHA}}
\end{figure}

The \ac{HIOA} shown in Figure~\ref{fig:heartCellHA} captures the
behaviour of the four stages of \ac{AP} shown in
Figure~\ref{fig:actionPotential}. Each stage of the \ac{AP} is captured
with so called locations in the \ac{HIOA}. The \ac{AP} of every node, in
the heart, evolves over time, this evolution of the \ac{AP} is captured
with three \acfp{ODE} evolving three individual variables: $v_{x}$,
$v_{y}$, and $v_{z}$, respectively. The overall voltage of the \ac{AP}
is then aggregated into a single variable $v$ as defined
in~\cite{YeESG08}. The aggregated \ac{AP} voltage $v$ of each node
effects its neighboring nodes and vice-versa. This effect of the
neighboring voltages is captured in the \ac{HIOA} via function
$g(\vec{v_{I}})$, where $\vec{v_{I}}$ represents the \emph{vector} of
voltages. Finally, the dynamic response of the action potential to
secondary excitation is captured by the variable~$\theta$.

For the first stage (\ac{RP}), the continuous dynamics are captured in
location $\mathbf{q_0}$. \textit{Ordinary Differential Equation} (ODEs),
for example \mbox{$\dot{v_{x}} = C_{1}v_{x}$}, capture the continuous
evolution of the voltage of the node in location $\mathbf{q_{0}}$ until
the so called location invariant
$v < V_{T} \wedge g(\vec{v_{I}}) \leq V_{T}$ holds.  Once the external
stimulus, $g(\vec{v_{I}})$, crosses some threshold $V_{T}$, indicated by
the predicate \mbox{$g(\vec{v_{I}} \geq V_{T})$} on the edge connecting
locations $\mathbf{q_{0}}$ and $\mathbf{q_{1}}$, the \ac{HIOA}
transitions to location $\mathbf{q_{1}}$
\textit{instantaneously}. During this transition, the variables (e.g,
$v^{\prime}_{x}$) are updated. These updated values of the variables are
used as initial conditions from which the variables evolve further via
the \acp{ODE} in the target location.

Once the voltage of the node ($v$) exceeds the threshold voltage
$V_{T}$, it begins its \ac{ERP} phase (captured by location
$\mathbf{q_2}$). The final location, $\mathbf{q_3}$, represents the
\ac{RRP} stage. The constants $C_1$ through to $C_{15}$, configure the
morphology of the \ac{AP}. Furthermore, constants $V_T$, $V_O$, and
$V_R$ determine the height of the \ac{AP} and transitions between
locations of the \ac{HIOA} that further modify the morphology.

% The voltage of any node in the heart effects its neighboring nodes and
% vice-versa. Hence, a means to communicate the voltages of the node to
% its neighbors is needed. We achieve this communication between nodes of
% the heart using the input and output variables and events available in
% the \ac{HIOA} framework. 


We formalise the \ac{HIOA} using Definition~\ref{def:ha}.

\begin{definition}
  % \label{def:1}
  A \acf{HIOA} \newline $\mathcal{H}$ =
  $\langle Loc, Edge, \Sigma_I, \Sigma_{EO}, \Sigma_{EI}, X_I, X_{EO},
  X_{EI}, \linebreak Init, Inv, Flow, Up, Jump\rangle$ where
  \begin{itemize}
  \item $Loc=\{l_1,..,l_n\}$ representing $n$ locations.
  \item $\Sigma_{E} = \Sigma_{EI} \cup \Sigma_{EO}$ is the set of
    external events. $\Sigma_{EI}$ and $\Sigma_{EO}$ are the set of
    external input and external output events,
    respectively. Furthermore,
    \mbox{$\Sigma_{EI} \cap \Sigma_{EO} = \emptyset$}.
  \item $\Sigma=\Sigma_{E} \cup \{\iSignal\}$ is set of event names
    comprising of external and the $true$ event, respectively.
  \item
    $Edge \subseteq Loc \times \mathcal{B}(\Sigma_{EI} \cup \{\tau\})
    \times 2^{\Sigma_{EO}} \times Loc$
    are the set of edges between locations.
  \item Three sets for the set of internal continuous variables, their
    rate of change and their updated values represented as follows:
    $X_I=\{x_1,.., x_m\}$, $\dot{X_I}=\{\dot{x_1},.., \dot{x_m}\}$, and
    $X_I'=\{x_{1}',.., x_{m}'\}$.
  \item The set of external continuous variables:
    \mbox{$X_E = X_{EI} \cup X_{EO}$}. Furthermore,
    \mbox{$X_{EI} \cap X_{EO} = \emptyset$}.
  \item $Init(l)$: Is a predicate whose free variables are from
    $X_{I}$. It specifies the possible valuations of these when the HIOA
    starts in $l$.
  \item $Inv(l)$: Is a predicate whose free variables are from
    \mbox{$X_I \cup X_{E}$} and it constrains these when the HIOA
    resides in $l$.
  \item $Flow(l)$: Is a predicate whose free variables are from
    $X_I \cup \dot{X_I}$ and it specifies the rate of change of these
    variables when the HIOA resides in $l$.
  \item $Up(l)$: Is a function that assigns to each variable in set
    $X_{EO}$ a valuation of the variables from the set $X_{I}$.
  \item $Jump(e)$: Is a function that assigns to the edge `$e$' a
    predicate whose free variables are from $X_I \cup X_I' \cup X_E$.
    This predicate specifies when the location switch using `$e$' is
    possible. It also specifies the updated values of the variables when
    this location switch happens.
  \end{itemize}
  \label{def:ha}
\end{definition}

For the \ac{HIOA} in Figure~\ref{fig:heartCellHA},
$Loc=\{q_{0},q_{1},q_{2},q_{3}\}$, $\Sigma_{EI} = \emptyset$,
$\Sigma_{EO}=\emptyset$, and
$\Sigma=\{\tau\}$. \mbox{$X_{I}=\{\theta,v_{x},v_{y},v_{z}\}$},
\mbox{$\dot{X_{I}}=\{\dot{\theta},\dot{v_{x}},\dot{v_{y}},\dot{v_{z}}\}$},
\mbox{$X^{\prime}_{I}=\{\theta',{v'_{x}},{v'_{y}},\dot{v'_{z}}\}$}.
$X_{EI}=\{v_{i}| \forall v_{i} \in \vec{v_{I}}\}$. $X_{EO}=\{v\}$. An
example of initial predicate is:
$Init(l): v_{x}=0 \wedge v_{y}=0 \wedge v_{z}=0 \wedge \theta=0$. An
example edge connecting locations $q_{0}$ and $q_{1}$ is represented
with the tuple: $(q_{0},\tau,\emptyset,q_{1})$. The example of invariant
($Inv$) predicate on location $q_{0}$ is:
$v < V_{T} \wedge g(\vec{v_{I}}) \leq V_{T}$. The $Jump$ condition on
the edge connecting locations $q_{0}$ and $q_{1}$ is enabled when the
predicate \mbox{$g(\vec{v_{I}} \geq V_{T})$} is true, and the updated
variables are as shown in Figure~\ref{fig:heartCellHA}. An example of
the $Up$ predicate is the function $v = v_{x} - v_{y} + v_{z}$ in all
locations in Figure~\ref{fig:heartCellHA}. Finally, the $Flow$ predicate
in location $q_{0}$ are the set of \acp{ODE}, in location
$\mathbf{q_{0}}$ in Figure~\ref{fig:heartCellHA}.

% \begin{definition}[Hybrid Automata]
%   \label{def:ha}
%   $HIOA = \langle Loc, X, \Sigma, Init, Inv, Flow, Jump \rangle$ where
%   %   Locations
%   $Loc=\{q_0,q_1,q_2,q_3\}$ is a set of locations.
%   $X=\{v,v_x,v_y,v_z,\theta\}$ is a set of continuous variables.
%   $\Sigma=\Sigma_I \cup \Sigma_O \cup \{\iSignal\}$ is a set of input,
%   output, and internal events.  $Init=\{(q_0,v): v < V_T\}$ is the
%   initial condition.  $Inv=\{(q_3,v): v > V_R\}$ is the invariant.
%   $Flow=\{(q_0,v_x): C_1 v_x\}$ specifies the continuous flow rate.
%   $Jump=\{(q_n, q_m): signal, cond, update\}$: describes the discrete
%   transition from $q_n$ and $q_m$ which is enabled when $signal$ is
%   present and the $cond$ evaluates to true, perfoms and $update$
%   during this transition.  Further, to remove non-determinism, we must
%   ensure that for all locations there does not exist more than one
%   active $Jump$ transition.
% \end{definition}

% {\color{red} Given $m$ number of \acp{HIOA} in a network. The network
% can be described as $HIOA_1 || \dots || HIOA_m$.}

% \section{\acf{SHIOA}}
% \label{sec:SHA}

% Every \ac{HIOA} is converted into a so called \ac{SHIOA}, where the
% ODEs in the flow predicates are replaced by their equivalent witness
% functions $Switness$. The witness function computes the evolution of
% the continuous variables using the forward Euler approach with a
% \emph{constant} step size. Before giving the formal definition of the
% \ac{SHIOA} and describing the compilation procedure, we present the
% procedure to convert each ODE into its equivalent witness function,
% this is the step 1 (c.f. Figure~\ref{fig:overview}) in our modular
% code generation framework.

% \renewcommand{\algorithmiccomment}[1]{// #1}
% \renewcommand{\algorithmicrequire}{\textbf{Input:}}
% \renewcommand{\algorithmicensure}{\textbf{Output:}}
% \begin{algorithm}[t!]
%   \begin{algorithmic}[1]
%     \REQUIRE Network of HIOA has \ENSURE Network of SHIOA shas
%     \FORALL{$ha \in has$} \label{alg:HAsToSHAs:allHAs}
%     \FORALL{$loc \in ha$} \label{alg:HAsToSHAs:allLocs}
%     \FORALL{$ode \in loc$} \label{alg:HAsToSHAs:allODEs} \STATE
%     $eq \leftarrow create\_euler(ode)$
%     \label{alg:HAsToSHAs:createEuler} \STATE $ode \leftarrow
%     eq$ \label{alg:HAsToSHAs:assignOde}
%     \ENDFOR
%     \ENDFOR
%     \ENDFOR
%     \RETURN this
%   \end{algorithmic}
%   \caption{The algorithm to generate a Network of \acp{SHIOA} from a
%   Network of \acp{HIOA}}
%   \label{alg:HAsToSHAs}
% \end{algorithm}

% The procedure for performing this conversion across the entire network
% of \acp{HIOA} to generate a network of \acp{SHIOA} is shown in
% Algorithm~\ref{alg:HAsToSHAs}. The algorithm needs to visit every
% \ac{HIOA} in the network, every location within each \ac{HIOA}, and
% every \ac{ODE} in each location (lines~\ref{alg:HAsToSHAs:allHAs} -
% \ref{alg:HAsToSHAs:allODEs}). For each of these \acp{ODE} the forward
% Euler equivalent is created in the same manner as described earlier
% (line~\ref{alg:HAsToSHAs:createEuler}). Each \ac{ODE} is eventually
% replaced (line~\ref{alg:HAsToSHAs:assignOde}) by its equivalent
% numerical solution in order to form the network of \acp{SHIOA}.

% \subsection{Definition of \acf{SHIOA}}
% \label{sec:defSHA}

% \begin{figure}
%   \centering \begin{tikzpicture}[->,>=stealth',shorten >=1pt,auto,
node distance=4cm,
semithick,scale=0.9, transform shape]
\tikzstyle{every state}=[rectangle,rounded corners,
 minimum height = 1.9cm, text width=3.35cm, text centered, fill=blue!20,draw=none,text=black, draw,line width=0.3mm]

\node[state, 
label={[shift={(0,0)}]$v < V_T \wedge g(\vec{v_{I}}) \leq V_T$}, 
label={[shift={(-1.5,-2.5)}]\large $ \mathbf{q_0}$ }]
(Q0)  {$v^\prime_x = v_x + \delta C_{1} v_x$ \\ $v^\prime_y = v_y + \delta C_{2} v_y$ \\ $v^\prime_z = v_z + \delta C_{3} v_z$ \\ $v^\prime=v^\prime_x - v^\prime_y + v^\prime_z$};

%entry
\draw[<-, dashed](Q0.150) -- node[above, shift={(0.3,0.4)}] {\footnotesize initial} ++(0cm,1cm);




\node[state, 
label={[shift={(0,0)}]$v < V_T \wedge g(\vec{v_{I}}) > 0$}, 
label={[shift={(-1.5,-2.5)}]\large $\mathbf{q_1}$   }]
(Q1) [node distance=6.3cm, right of=Q0] {\small{ $v^\prime_x {=} v_x {+} \delta (C_{4} v_x {+} C_{7} g(\vec{v_{I}}))$ \\ $v^\prime_y {=} v_y {+} \delta (C_{5} v_y {+} C_{8} g(\vec{v_{I}}))$ \\ $v^\prime_z {=} v_z {+} \delta (C_{6} v_z {+} C_{9} g(\vec{v_{I}}))$ \\ $v^\prime=v^\prime_x - v^\prime_y + v^\prime_z$}};

\node[state, 
label={[shift={(0.2,-2.5)}]$v < V_O - 80.1 \sqrt{\theta}$}, 
label={[shift={(-1.5,-2.5)}]\large $\mathbf{q_2}$  }] 
(Q2) [below of=Q1] {$v^\prime_x = v_x + \delta C_{10} v_x$ \\ $v^\prime_y = v_y + \delta C_{11} v_y$ \\ $v^\prime_z = v_z + \delta C_{12} v_z$ \\ $v^\prime=v^\prime_x - v^\prime_y + v^\prime_z$};

\node[state, 
label={[shift={(0,-2.5)}]$v > V_R$}, 
label={[shift={(-1.5,-2.5)}]\large $\mathbf{q_3}$  }]
(Q3) [below of=Q0] {$v^\prime_x = v_x + \delta C_{13} v_x f(\theta)$ \\ $v^\prime_y = v_y + \delta C_{14} v_y f(\theta)$ \\ $v^\prime_z = v_z + \delta C_{15} v_z$ \\ $v^\prime=v^\prime_x - v^\prime_y + v^\prime_z$};

\path[->] (Q0.10) edge node[align=center] {
	\footnotesize $\{g(\vec{v_{I}}) \geq V_T\}$, \\
	\footnotesize $\{\emptyset\}$, \\
	\footnotesize $\left\{\begin{matrix} v^\prime_x = 0.3v \\ v^\prime_y = 0.0v \\ v^\prime_z = 0.7v \\ \theta^\prime = v / V_T \end{matrix}\right\}$
} (Q1.170);

\path[->] (Q1.190) edge node[align=center] {
	\footnotesize $\{g(\vec{v_{I}}) \leq 0\}$, \\
	\footnotesize $\{v < V_T\}$, \\
	\footnotesize $\left\{\begin{matrix} v^\prime_x = v_x \\ v^\prime_y = v_y \\ v^\prime_z = v_z \end{matrix}\right\}$
} (Q0.350);

\path[->] (Q1) edge node[align=center] {
	\footnotesize $\{\tau\}$, \\
	\footnotesize $\{v \geq V_T\}$, \\
	\footnotesize $\left\{\begin{matrix} v^\prime_x = v_x \\ v^\prime_y = v_y \\ v^\prime_z = v_z \end{matrix}\right\}$
} (Q2);

\path[->] (Q2) edge node[align=center] {
	\footnotesize $\{\tau\}$, \\
	\footnotesize $\{v \geq V_O - 80.1\sqrt{\theta}\}$, \\
	\footnotesize $\left\{\begin{matrix} v^\prime_x = v_x \\ v^\prime_y = v_y \\ v^\prime_z = v_z \end{matrix}\right\}$
} (Q3);

\path[->] (Q3) edge node[align=center, shift={(1.8,0)}] {
	\footnotesize $\{\tau\}$, \\
	\footnotesize $\{v \leq V_R\}$, \\
	\footnotesize $\left\{\begin{matrix} v^\prime_x = v_x \\ v^\prime_y = v_y \\ v^\prime_z = v_z \end{matrix}\right\}$
} (Q0);

\end{tikzpicture}
%   \caption{\acf{SHIOA} of a heart node \label{fig:heartCellSHA}}
% \end{figure}

% A \ac{SHIOA} is an abstraction of the corresponding \ac{HIOA}. It
% inherits all the components of a \ac{HIOA} except that the ODEs in the
% $Flow$ predicates are replaced by their individual witness functions
% $Switness$. The
% $Switness(l,\delta,\vec{v}_{l,X_{I}})=\vec{v}^{\prime}_{l,X_{I}}$,
% function returns the new values of continuous variables in location
% $l$ after one iteration of step size $\delta$ given the current
% valuation ($\vec{v}^{\prime}_{l,X_{I}}$) of continuous variables in
% $X_{I}$. We now formalise the \ac{SHIOA} using
% Definition~\ref{def:sha} and illustrate it using
% Figure~\ref{fig:heartCellSHA}.

% \begin{definition}
%   Given a well formed HIOA $\mathcal{H}$ =
%   $\langle Loc, Edge, \Sigma_I, \Sigma_{EO}, \Sigma_{EI}, X_I, X_{EO},
%   X_{EI}, \linebreak Init, Inv, Flow, Jump\rangle$
%   a SHIOA corresponding to this HIOA is $\mathcal{A}$ =
%   $\langle Loc, Edge, \Sigma_I, \Sigma_{EO}, \Sigma_{EI}, X_I,
%   X_{EO},\linebreak X_{EI}, Init, Inv, Witness, Jump, Step\rangle$
%%   where:
%   \begin{itemize}
%   \item $Step = \delta$ ($\delta \in \mathbb{R}^+$) is the duration of
%     the synchronous instant.
%   \item
%     $Switness: Loc \times \mathbb{R} \times \mathbb{R}^{\|X_{I}\|}
%     \rightarrow \mathbb{R}^{\|X_{I}\|}$:
%     is the Euler function(s) that returns the new values of all the
%     continuous variables for a given location $l$. It also takes as
%     input the time step $\delta$ and the current values of all
%     variables in set $X_{I}$.
%   \end{itemize}
%   \label{def:sha}
% \end{definition}

% {\color{red} Given $m$ number of \acp{SHIOA} in a network. The network
% can be described as $SHIOA_1 || \dots || SHIOA_m$.}

% \subsection{Deterministic semantics of \ac{SHIOA}}
% \label{sec:DTTS}
% Deterministic semantics of a \ac{SHIOA} is provided as a \acf{DTTS} in
% Definition~\ref{def:dtts}. We assume that all transitions of a
% \acf{DTTS} trigger relative to the ticks of the logical clock of the
% synchronous program.
% 
% \begin{definition}
%   The semantics of a \newline
%   $SHIOA = \langle Loc, \Sigma, Init, Inv, Switness, Jump, Step
%   \rangle$
%   is a $DTTS = \langle Q, Q^0, \Sigma, \rightarrow \rangle$ where
%   
%   \begin{itemize}
%   \item The state-space is $Q$, where any state is of the form
%     $(l, v, i, k)$ where $l$ is a location, $i$ is the initial
%     valuation of the variables when execution begins in the location
%     and $v$ is the valuation at the $k$-th instant.
%   \item $Q^0 \subseteq Q$ where every $q^0 \in Q^0$ is of the form
%     $(l, v^0, i, k)$ such that $v$ satisfies $Init(l)$.
%   \item Transitions are of two types:
%     \begin{itemize}
%     \item \emph{Inter-location transitions} that lead to a change in
%       location: These are of the form
%       $(l, v, i, k) \stackrel{\sigma} \rightarrow (l', v', i', 0)$ if
%       $(l, v, i, k) \in Q$, $(l', v', i', 0) \in Q$,
%       $e=(l \stackrel{\sigma} \rightarrow l') \in Edge$ and $(v, v')$
%       satisfy $Jump(e)$.
%     \item \emph{Intra-location transitions} made during the execution
%       in a given mode / location: These are of the form
%       $(l, v, i, k) \rightarrow (l, v', i, k+1)$ if
%       $(l, v, i, k) \in Q$, $(l, v', i, k+1) \in Q$, $(v, v')$ satisfy
%       $Inv(l)$, $Switness(l,k,\delta,i)=v$ and
%       $Switness(l,k+1,\delta,i)=v'$.
%     \end{itemize}
%     
%			\item Restriction: Inter-location transitions
%     always have \emph{higher} priority over {Intra-location
%     transitions}. This avoids the non-determinism.
%   \end{itemize}
%   \label{def:dtts}
% \end{definition}

% \subsection{Generation of a network of \ac{SHIOA}}
% \label{sec:shaGeneration}

% This section describes step $1$ of the compilation process from
% Figure~\ref{fig:overview}. This step involves converting all of the
% \acp{ODE} in each of the \acp{HIOA} into their forward Euler
% equivalent methods.

%%% Local Variables:
%%% mode: latex
%%% TeX-master: "../DATE2016_codegen"
%%% End:
