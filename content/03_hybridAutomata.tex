\section{Hybrid Automata}
\label{sec:HA}



\acf{HA} is ideal for describing 
Cyber-physical systems. It is very expressive and
some constraints need to be imposed s.t. it is possible to  
generate synthesisable  code~[X]. In this section, 
we define a restricted \ac{HA} and then propose a discrete abstraction
called \acf{SHA} which is based on the
synchronous approach~\cite{benveniste03}.
Finally, we propose a deterministic semantics for \ac{SHA}.



\subsection{Example of \acf{HA} }

\begin{figure}
\centering
\begin{tikzpicture}[->,>=stealth',shorten >=1pt,auto,
node distance=4cm,
semithick,scale=0.9, transform shape]
\tikzstyle{every state}=[rectangle,rounded corners,
 minimum height = 1.2cm, text width=3.3cm, text centered, fill=blue!20,draw=none,text=black, draw,line width=0.3mm]

\node[state, 
label={[shift={(0,0)}]$v \leq V_T$}, 
label={[shift={(-1.5,0)}]\large $ \mathbf{q_0}$ }]
(Q0)  {$\dot{v_x}=a^0_x v_x$ \\ $\dot{v_y}=a^0_y v_y$ \\ $\dot{v_z}=a^0_z v_z$ \\ $v=v_x - v_y + v_z$};

%entry
\draw[<-, dashed](Q0.130) -- node[above, shift={(0,0.4)}] {\footnotesize initial} ++(0cm,1cm);




\node[state, 
label={[shift={(0,0)}]$v \leq v_T$}, 
label={[shift={(-1.5,0)}]\large $\mathbf{q_1}$   }]
(Q1) [node distance=6.3cm, right of=Q0] {$\dot{v_x}=a^1_x v_x + \beta_x g(\vec{v})$ \\ $\dot{v_y}=a^1_y v_y + \beta_y g(\vec{v})$ \\ $\dot{v_z}=a^1_z v_z + \beta_z g(\vec{v})$ \\ $v=v_x - v_y + v_z$};

\node[state, 
label={[shift={(0,-2.5)}]$v \leq V_O - 80.1 \sqrt{\theta}$}, 
label={[shift={(-1.5,0)}]\large $\mathbf{q_2}$  }] 
(Q2) [below of=Q1] {$\dot{v_x}=a^2_x v_x$ \\ $\dot{v_y}=a^2_y v_y$ \\ $\dot{v_z}=a^2_z v_z$ \\ $v=v_x - v_y + v_z$};

\node[state, 
label={[shift={(0,-2.5)}]$v \geq V_R$}, 
label={[shift={(-1.5,0)}]\large $\mathbf{q_3}$  }]
(Q3) [below of=Q0] {$\dot{v_x}=a^3_x v_x f(\theta)$ \\ $\dot{v_y}=a^3_y v_y f(\theta)$ \\ $\dot{v_z}=a^3_z v_z$ \\ $v=v_x - v_y + v_z$};

\path[->] (Q0.10) edge node[align=center] {
	\footnotesize $\{g(\vec{v}) > V_T\}$, \\
	\footnotesize $\{\emptyset\}$, \\
	\footnotesize $\left\{\begin{matrix} v^\prime_x = 0.3v \\ v^\prime_y = 0.0v \\ v^\prime_z = 0.7v \\ \theta^\prime = v / V_T \end{matrix}\right\}$
} (Q1.170);

\path[->] (Q1.190) edge node[align=center] {
	\footnotesize $\{g(\vec{v}) \leq 0\}$, \\
	\footnotesize $\{v < V_T\}$, \\
	\footnotesize $\left\{\begin{matrix} v^\prime_x = v_x \\ v^\prime_y = v_y \\ v^\prime_z = v_z \end{matrix}\right\}$
} (Q0.350);

\path[->] (Q1) edge node[align=center] {
	\footnotesize $\{\tau\}$, \\
	\footnotesize $\{v \geq V_T\}$, \\
	\footnotesize $\left\{\begin{matrix} v^\prime_x = v_x \\ v^\prime_y = v_y \\ v^\prime_z = v_z \end{matrix}\right\}$
} (Q2);

\path[->] (Q2) edge node[align=center] {
	\footnotesize $\{\tau\}$, \\
	\footnotesize $\{v \geq V_O - 80.1\sqrt{\theta}\}$, \\
	\footnotesize $\left\{\begin{matrix} v^\prime_x = v_x \\ v^\prime_y = v_y \\ v^\prime_z = v_z \end{matrix}\right\}$
} (Q3);

\path[->] (Q3) edge node[align=center, shift={(1.8,0)}] {
	\footnotesize $\{\tau\}$, \\
	\footnotesize $\{v \leq V_R\}$, \\
	\footnotesize $\left\{\begin{matrix} v^\prime_x = v_x \\ v^\prime_y = v_y \\ v^\prime_z = v_z \end{matrix}\right\}$
} (Q0);

\draw[->](Q0)--(Q2);

\end{tikzpicture}
\caption{\acf{HA} of a heart cell \label{fig:heartCellHA}}
\end{figure}


For the four stages of action potential  shown in Figure~\ref{fig:actionPotential}, the \ac{HA} is
shown in Figure~\ref{fig:heartCellHA}.
The first stage (\ac{RP}), the continuous behaviour is captured 
by location~$q_0$. (ADD MORE DESCRIPTION ABOUT THE CELL).

For our hybrid automata model, we need to be able to 
describe  input ($?s$), output ($!s$) and internal signals ($\iSignal$). 
Further,
the signals can be classified as either 
\emph{pure} signals (absent, or present) or
\emph{valued} signals (absent, or has a value).
We now formalise
the \ac{HA} using Definition~\ref{def:ha} 
and illustrate using Figure~\ref{fig:heartCellHA}.\newline



\begin{definition}[Hybrid Automata]
	\label{def:ha}
	 $HA = \langle Loc, X, \Sigma, Init, Inv, Flow,$ $ Jump \rangle$ where
	%Locations
	$Loc=\{q_0,q_1,q_2,q_3\}$ is a set of locations.
	$X=\{v,v_x,v_y,v_z\}$ is a set of continuous variables.
	$\Sigma=\Sigma_I \cup \Sigma_O \cup \{\iSignal\}$ is a set of
	 input, output, and internal events.
	$Init=\{(q_0,v): v\leq V_T\}$ is the initial condition.
	$Inv=\{(q_3,v): v\geq V_R\}$ is the invariant.
	$Flow=\{(q_0,v_x): a_{x}^{0} v_x\}$ specifies the continuous flow rate.
	$Jump=\{(q_n, q_m): signal, cond, update\}$: describes the discrete
	transition from $q_n$ and $q_m$ is enabled when $signal$ is 
	present and the $cond$ evaluates to true and $update$
	some variable during this transition.
	Further, to reduce non-determinism, 
	for all locations, must does not exist more than one 
	active $Jump$ transition.
\end{definition}

\begin{definition}[Network of  Hybrid Automata]
	\label{def:nha}
	Let $m$  be the number of \acp{HA} in the network.
	The network can be defined as $HA_1 || \dots || HA_m$
\end{definition}

\section{Synchronous Hybrid Automata}
\label{sec:SHA}
 
Before we present step~1 of Figure~\ref{fig:overview},
in Section~\ref{sec:defSHA} define the discrete form of \ac{HA}, 
called \acf{SHA} and present its deterministic semantics in 
Section~\ref{sec:DTTS}

\subsection{Definition of \acf{SHA}}
\label{sec:defSHA}
A \ac{SHA} is a discreatised  abstraction of the corresponding \ac{HA}. 
It inherits all the
components of a \ac{HA} except that $Flow$ predicates are replaced by a
composite witness function $Switness$.
$Switness(l,k, \delta, i)=v_{k,l,i}$, which returns the evaluation of
the witness function in location $l$ at time step $k$.
We now formalise
the \ac{SHA} using Definition~\ref{def:sha} and illustrate using Figure~\ref{fig:heartCellSHA}.



\begin{definition}
	Given a 
	$HA = \langle Loc, X, \Sigma, Init,$ $ Inv, Flow, Jump \rangle$ a \ac{SHA} corresponding to
	this \ac{HA} is \newline
	$SHA = \langle Loc,  \Sigma, Init, Inv, Switness, Jump, Step \rangle$ where:
	\begin{itemize}
		\item $Step = \delta \in \mathbb{R}^+$: This specifies the duration of the synchronous instant,  duration between 
		two samples.
		\item
		$Switness: Loc \times \mathbb{N} \times \mathbb{R} \times
		\mathbb{R}^m \rightarrow \mathbb{R}^m$:
		is the witness function that returns the valuation of all the
		continuous variables at any valid time step $k$ in a given location
		$l$. It also takes as input the time step $\delta$ and the initial
		value from which execution begins in the location.
	\end{itemize}
	\label{def:sha}
\end{definition}


\begin{figure}
	\centering
	\input{figures/cellSHA}
	\caption{\acf{SHA} of the heart cell \label{fig:heartCellSHA}}
\end{figure}

\subsection{Deterministic semantics of \ac{SHA}}
\label{sec:DTTS}
Deterministic semantics of a \ac{SHA} is provided as a \acf{DTTS} in
Definition~\ref{def:dtts}. We assume that all transitions of a \acf{DTTS}
trigger relative to the ticks of the logical clock of the synchronous
program.

\begin{definition}
	The semantics of a \newline
	$SHA = \langle Loc,  \Sigma, Init, Inv, Switness, Jump, Step \rangle$
	 is a 
	$DTTS = \langle Q, Q^0, \Sigma, \rightarrow \rangle$ where
	
	\begin{itemize}
		\item The state-space is $Q$, where any state is of the form
		$(l, v, i, k)$ where $l$ is a location, $i$ is the initial valuation
		of the variables when execution begins in the location and $v$ is
		the valuation at the $k$-th instant.
		\item $Q^0 \subseteq Q$ where every $q^0 \in Q^0$ is of the form
		$(l, v^0, i, k)$ such that $v$ satisfies $Init(l)$.
		\item Transitions are of two types:
		\begin{itemize}
			\item \emph{Inter-location transitions} that lead to a change in location:
			These are of the form
			$(l, v, i, k) \stackrel{\sigma} \rightarrow (l', v', i', 0)$ if
			$(l, v, i, k) \in Q$, $(l', v', i', 0) \in Q$,
			$e=(l \stackrel{\sigma} \rightarrow l') \in Edge$ and $(v, v')$
			satisfy $Jump(e)$.
			\item \emph{Intra-location transitions} made during the execution in
			a given mode / location: These are of the form
			$(l, v, i, k) \rightarrow (l, v', i, k+1)$ if
			$(l, v, i, k) \in Q$, $(l, v', i, k+1) \in Q$, $(v, v')$ satisfy
			$Inv(l)$, $Switness(l,k,\delta,i)=v$ and $Switness(l,k+1,\delta,i)=v'$.
		\end{itemize}
		
			\item Restriction: Inter-location transitions always have \emph{higher} priority
			over {Intra-location transitions}. This avoids the non-determinism.
	\end{itemize}
	\label{def:dtts}
\end{definition}

\subsection{Generation of a network of \ac{SHA}}
\label{sec:shaGeneration}

This section describes step $1$ of the compilation process from Figure~\ref{fig:compilingHA}.
This step involves converting all of the \acp{ODE} in each of the \acp{HA} into their forward Euler equivalent methods.

Given the \ac{ODE} for $v_x$ in state $\mathbf{q_1}$ of the Heart Cell \ac{HA} defined in Figure~\ref{fig:cellHA} (reproduced in Equation~\ref{eq:ode}), the forward Euler method is applied on each iteration in order to calculate $v_x$ numerically.
The forward Euler equivalent for this ODE is shown in Equation~\ref{eq:euler_equiv} where $\delta$ is the time-step size and $v^\prime_x$ represents the value of $v_x$ after the current iteration.

\begin{equation}
\dot{v_x} = a^1_x v_x + \beta_x g(\vec{v})
\label{eq:ode}
\end{equation}

\begin{equation}
v^\prime_x = v_x + \delta * (a^1_x v_x + \beta_x g(\vec{v}))
\label{eq:euler_equiv}
\end{equation}

\renewcommand{\algorithmiccomment}[1]{// #1}
\renewcommand{\algorithmicrequire}{\textbf{Input:}}
\renewcommand{\algorithmicensure}{\textbf{Output:}}
\begin{algorithm}[t!]
	\begin{algorithmic}[1]
		\REQUIRE Network of HA has 
		\ENSURE Network of SHA shas
		\FORALL{$ha \in has$} \label{alg:HAsToSHAs:allHAs}
		\FORALL{$loc \in ha$}  \label{alg:HAsToSHAs:allLocs}
		\FORALL{$ode \in loc$} \label{alg:HAsToSHAs:allODEs}
		\STATE $eq \leftarrow create\_euler(ode)$ \label{alg:HAsToSHAs:createEuler}
		\STATE $ode \leftarrow eq$ \label{alg:HAsToSHAs:assignOde}
		\ENDFOR
		\ENDFOR
		\ENDFOR
		\RETURN this
	\end{algorithmic}
	\caption{The algorithm to generate a Network of \acp{SHA} from a Network of \acp{HA}}
	\label{alg:HAsToSHAs}
\end{algorithm}

The procedure for performing this conversion across the entire network of \acp{HA} to generate a network of \acp{SHA} is shown in Algorithm~\ref{alg:HAsToSHAs}.
The algorithm needs to visit every \ac{HA} in the network, every location within each \ac{HA}, and every \ac{ODE} in each location (lines~\ref{alg:HAsToSHAs:allHAs} - \ref{alg:HAsToSHAs:allODEs}).
For each of these \acp{ODE} the forward Euler equivalent is created in the same manner as described earlier (line~\ref{alg:HAsToSHAs:createEuler}).
Each \ac{ODE} is eventually replaced (line~\ref{alg:HAsToSHAs:assignOde}) by its equivalent numerical solution in order to form the network of \acp{SHA}.
