\begin{abstract}
  We study the problem of modular code generation for emulating the
  electrical conduction system of the heart, which is essential for the
  validation of implantable devices such as pacemakers. For developing
  such high fidelity models, it is essential to consider the operation
  of hundreds, if not millions of conduction elements, called
  \emph{nodes} of the heart. Published results so far, however, have
  considered a maximum of 33~nodes, modelled as \acf{HIOA}. The behaviour
  of this model is captured using the well known commercial tool
  \simulink. The proposed approach is limiting due to the lack of model
  fidelity %as well as the lack of modelling the re-entrant nature 
  of the  conduction system.

  In this paper, we have developed a radically new approach to tackle
  the aforementioned drawbacks. We first develop a semantic preserving
  modular compilation approach for a network of \ac{HIOA}, by proposing
  to translate them to a network of so called \ac{SHIOA}.  We then
  demonstrate that a delayed synchronous composition of the cardiac
  nodes enables%: %(1) capturing the re-entrant nature of the conduction
  %system and (2) 
  modular code generation that is both semantic
  preserving and efficient. In addition to the above example, we have
  developed several examples from other domains to compare \simulink and
  the developed tool called \textit{Piha}. The results show that we are
  able to generate code that is on an average 54\% smaller in binary
  size while executing 9.8~times faster compared to \simulink.  We have
  also compared the scalability of the proposed approach relative to
  \simulink to prove its relative superiority.

  \ignore{Real-time emulation of a plant is often desirable to
    facilitate the testing of controllers under realistic conditions.
    Plants which exhibit continuous dynamics can be well modelled
    through the use of \acf{HIOA}, and so code generation from \ac{HIOA}
    that is suitable for real-time emulation is desirable.  In the case
    of plants which are modelled by a network of \acp{HIOA}, such code
    generation must be able to emulate the entire overall system through
    parallel composition.

    In this paper, we propose a new tool (named \ourTool) for the
    modular code generation from \ac{HIOA} that is amenable to use in
    both the real-time emulation or simulation of plants.  We illustrate
    the suitability of this approach through the running example of a
    human heart, where real-time emulation is desired for the
    verification of cardiac pacemakers.  We compare the proposed
    approach to \simulink, a tool commonly used in the modelling of
    plants, to show that we are able to generate code that is on average
    54\% smaller while executing 9.8 times faster.  We finish by showing
    the scalability of our approach in order to illustrate its potential
    in allowing real-time emulation of complex \ac{HIOA} networks.}
\end{abstract}