
\section{Parallel Composition}
\label{sec:composition}

%\begin{figure}
%  \centering
%  \begin{tikzpicture}[->,>=stealth',shorten >=1pt,auto,node distance=4cm,
semithick,scale=0.9, transform shape]
\tikzstyle{every state}=[circle,rounded corners,
minimum height = 1.2cm, text width=1.2cm, text
centered, fill=blue!20,draw=none,text=black,
draw,line width=0.3mm]

\node[state] (IO)  { $I/O$ };

\path[<-, dashed] (IO.90) edge node[above, align=center, shift={(0.6,0)}] {
	initial
} ++(0cm,1cm);

\node[state]
(Cell1) [below right=0.37cm and 0.9cm of IO] 	 {$Cell1$};

\node[state]
(Cell2) [below left=1.28cm and -0.33cm of Cell1] 	 {$Cell2$};

\node[state]
(Cell33) [below left=0.37cm and 0.9cm of IO] 	 {$Cell33$};

\node[state]
(Cell32) [below right=1.28cm and -0.33cm of Cell33] 	 {$Cell32$};


\path[->] (IO.0) edge[out=0,in=108] (Cell1.108);

\path[->] (Cell1.-72) edge[out=-72,in=36] (Cell2.36);

\path[->, dotted] (Cell2.-144) edge[out=-144,in=-36] (Cell32.-36);

\path[->] (Cell32.-216) edge[out=-216,in=-108] (Cell33.-108);

\path[->] (Cell33.-288) edge[out=-288,in=-180] (IO.-180);

\end{tikzpicture}
%  \caption{Synchronous composition of multiple heart
%    cells \label{fig:heartCellComposition}}
%\end{figure}

In order to compose each of the \acp{FSM} together, we take inspiration
from the concepts of Synchronous Languages such as SL~\cite{SlLanguage}.
The concept of Ticks and Reactions are carried over, whereby each
\ac{FSM} performs only a single step (``tick'') until all \acp{FSM} have
completed a step (``reaction'') and the process can repeat.  In order to
deal with data dependencies, we also implement the concept of \emph{pre}
whereby the value of all inputs to each \ac{FSM} are not updated with
new values until the end of that reaction.  The concept of \emph{pre}
also enables \ourTool to simplify the process of handling cyclic
\acp{ODE} (an issue that other tools do not
consider~\cite{kim2003modular}) by causing the dependencies to refer to
previous, rather than current, values.

The implementation of this in \ourTool is done by creating a round-robin
scheduler which executes one tick of each \ac{FSM} in series, followed
by I/O synchronisation at the end of each reaction.  Such I/O
synchronisation deals with the emission of all outputs from the system,
the intake of all inputs to the system, and the transfer of signals
between \acp{FSM} in the network.  This idea is illustrated in
Figure~\ref{fig:heartCellComposition}.  This concept is amenable to
further scheduling algorithms that need not be sequential in nature.
\ourTool can be extended to support parallel computation by executing
different \acp{FSM} on separate threads, with synchronisation between
threads occurring at the end of each reaction.