
\section{Parallel Composition}
\label{sec:composition}

%\begin{figure}
%  \centering
%  \input{figures/cellComposition}
%  \caption{Synchronous composition of multiple heart
%    cells \label{fig:heartCellComposition}}
%\end{figure}

In order to compose each of the \acp{FSM} together, we take inspiration
from the concepts of Synchronous Languages such as SL~\cite{SlLanguage}.
The concept of Ticks and Reactions are carried over, whereby each
\ac{FSM} performs only a single step (``tick'') until all \acp{FSM} have
completed a step (``reaction'') and the process can repeat.  In order to
deal with data dependencies, we also implement the concept of \emph{pre}
whereby the value of all inputs to each \ac{FSM} are not updated with
new values until the end of that reaction.  The concept of \emph{pre}
also enables \ourTool to simplify the process of handling cyclic
\acp{ODE} (an issue that other tools do not
consider~\cite{kim2003modular}) by causing the dependencies to refer to
previous, rather than current, values.

The implementation of this in \ourTool is done by creating a round-robin
scheduler which executes one tick of each \ac{FSM} in series, followed
by I/O synchronisation at the end of each reaction.  Such I/O
synchronisation deals with the emission of all outputs from the system,
the intake of all inputs to the system, and the transfer of signals
between \acp{FSM} in the network.  This idea is illustrated in
Figure~\ref{fig:heartCellComposition}.  This concept is amenable to
further scheduling algorithms that need not be sequential in nature.
\ourTool can be extended to support parallel computation by executing
different \acp{FSM} on separate threads, with synchronisation between
threads occurring at the end of each reaction.