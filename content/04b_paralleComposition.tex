\section{Parallel Composition}
\label{sec:composition}

\subsection{Synchronous Parallel}
\label{sec:synchronousParallel}

%\begin{figure}
%  \centering
%  \begin{tikzpicture}[->,>=stealth',shorten >=1pt,auto,node distance=4cm,
semithick,scale=0.9, transform shape]
\tikzstyle{every state}=[circle,rounded corners,
minimum height = 1.2cm, text width=1.2cm, text
centered, fill=blue!20,draw=none,text=black,
draw,line width=0.3mm]

\node[state] (IO)  { $I/O$ };

\path[<-, dashed] (IO.90) edge node[above, align=center, shift={(0.6,0)}] {
	initial
} ++(0cm,1cm);

\node[state]
(Cell1) [below right=0.37cm and 0.9cm of IO] 	 {$Cell1$};

\node[state]
(Cell2) [below left=1.28cm and -0.33cm of Cell1] 	 {$Cell2$};

\node[state]
(Cell33) [below left=0.37cm and 0.9cm of IO] 	 {$Cell33$};

\node[state]
(Cell32) [below right=1.28cm and -0.33cm of Cell33] 	 {$Cell32$};


\path[->] (IO.0) edge[out=0,in=108] (Cell1.108);

\path[->] (Cell1.-72) edge[out=-72,in=36] (Cell2.36);

\path[->, dotted] (Cell2.-144) edge[out=-144,in=-36] (Cell32.-36);

\path[->] (Cell32.-216) edge[out=-216,in=-108] (Cell33.-108);

\path[->] (Cell33.-288) edge[out=-288,in=-180] (IO.-180);

\end{tikzpicture}
%  \caption{Synchronous composition of multiple heart
%    cells \label{fig:heartCellComposition}}
%\end{figure}

In order to compose each of the \acp{FSM} together, we take inspiration
from the concepts of Synchronous Languages such as SL~\cite{SlLanguage}.  The 
concept of Ticks and Reactions are carried over, whereby each \ac{FSM} performs 
only a single transition (``tick'') until all other \acp{FSM} have also 
completed a transition (``reaction'') and the process can repeat.  In order to
deal with data dependencies, we also implement the concept of \emph{pre}
whereby the value of all inputs to each \ac{FSM} are not updated with
new values until the end of each reaction.  This allows for the behaviour of 
the system to be agnostic to the scheduling order of the individual 
\acp{FSM}.  This concept also enables us to simplify the process of 
handling cyclic \acp{ODE} (an issue that other work such as 
\cite{kim2003modular} do not consider) by causing the dependencies to reference
previous, rather than current, values.

The implementation of this in \ourTool is done by creating a round-robin
scheduler which executes one tick of each \ac{FSM} (the \emph{Run Function} 
described in Section~\ref{sec:backendCodeGeneration}) in series, followed by an 
\emph{I/O Synchronisation Stage} at the end of each reaction.  Such I/O 
synchronisation deals with the emission of all outputs from the system, the 
intake of all inputs to the system, and the transfer of signals between 
\acp{FSM} in the network, the mapping of which is described in  
Section~\ref{sec:mapping}.  This process continues indefinitely to emulate the 
dynamics of the system.  In addition, when the system initially starts up it 
enters an \emph{Initialisation Stage} whereby the \emph{Initalisation Function} 
for each \ac{FSM} is run before continuing on to the round-robin loop mentioned 
above.

Due to the scheduling order agnosticism described earlier, this concept is 
amenable to further scheduling algorithms that need not be sequential in 
nature.  \ourTool can be extended to support parallel computation by executing 
different \acp{FSM} on separate threads (whether logical or physical), with 
synchronisation between threads occurring at the end of each reaction.

\subsection{Mapping between \acp{HIOA}}
\label{sec:mapping}
% Need to explain the mapping function \lambda : X_E -> X_E

